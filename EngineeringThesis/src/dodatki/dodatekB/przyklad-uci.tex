\chapter{Protokół UCI}
\label{ch:protokol-uci}

\section{Wykorzystane komendy}
\label{sec:wykorzystane-komendy}

\begin{table}[htb] \small
\centering
\caption{UCI - komunikacja GUI do silnika}
\label{tab:UCI_GUI_silnik}
\begin{tabularx}{\linewidth}{|p{.35\linewidth}|X|}\hline
\textbf{Komenda} & \textbf{Opis działania} \\ \hline\hline

\texttt{uci} &
Określenie używanego protokołu.
Po otrzymaniu komendy silnik powinien odpowiedzieć \texttt{id} oraz listą dostępnych opcji.\\ \hline

\texttt{setoption name~<name> value~<value>} &
Instrukcja zmiany wewnętrznego parametru silnika.
Nazwa parametru jest podawana przez silnik.
Wartość może być typu bool, int, String. \\ \hline

\texttt{position [startpos~|~<fen>] moves <moves>} &
Ustawienie pozycji zaczynając od startowej bądź z~podanego FEN.
Możliwe wykonanie dalszych ruchów przekazanych w formacie LAN.\\ \hline

\texttt{isready} &
Zapytanie, służące synchronizacji GUI z silnikiem.
Gdy silnik nie przetwarza poleceń powinien odpowiedzieć \texttt{uciok}.\\ \hline

\texttt{go wtime~<wtime> btime~<btime> winc~<winc> binc~<binc>} &
Instrukcja rozpoczęcia szukania ruchu.
Kolejne argumenty w milisekundach wyrażają: pozostały czas białego i czarnego, dodatkowy czas per ruch białego i czarnego.\\ \hline

\texttt{debug [on | off]} &
Zmiana trybu diagnostycznego silnika. \\ \hline

\texttt{quit} &
Kończy działanie silnika. \\ \hline

\end{tabularx}
\end{table}

\begin{table}[htb] \small
\centering
\caption{UCI - komunikacja silnika do GUI}
\label{tab:UCI_silnik_GUI}
\begin{tabularx}{\linewidth}{|p{.35\linewidth}|X|}\hline
\textbf{Komenda} & \textbf{Opis działania} \\ \hline\hline

\texttt{id [name | author] <value>} &
Instrukcja służąca identyfikacji silnika przez GUI. \\ \hline

\texttt{option name <name> type <type>} &
Wypisanie dostępnych opcji oraz typu ich działania. \\ \hline

\texttt{uciok} &
Wysyłany po \texttt{id} oraz \texttt{option}, aby potwierdzić poprawne wykonanie komedny \texttt{uci}. \\ \hline

\texttt{readyok} &
Odpowiedź na \texttt{isready}.
Synchronizuje GUI z silnikiem.\\ \hline

\texttt{info <value>} &
Przekazywanie GUI informacji nad aktualnym stanem obliczeń najlepszego ruchu.\\ \hline

\texttt{bestmove <move>} &
Odpowiedź na instrukcję \texttt{go}.
Zwraca ruch proponowany przez silnik.\\ \hline

\end{tabularx}
\end{table}

\section{Przykład użycia}
\label{sec:przyklad-uzycia}

\begin{table}[htb] \footnotesize
\centering
\caption{Przykład użycia UCI}
\label{tab:przyklad_uci}
\begin{tabularx}{\linewidth}{|>{\raggedright\arraybackslash}p{.51\linewidth}>{\raggedleft\arraybackslash}X|}\hline
\textbf{Output GUI} & \textbf{Output silnika} \\ \hline\hline

\texttt{uci} & \texttt{} \\ \hline
\texttt{} & \texttt{
    id name KobayashiMaru \linebreak
    id author Krzysztof Antoni Wiśniewski \linebreak
    option name OwnBook type check \linebreak
    uciok
} \\ \hline
\texttt{setoption name OwnBook value false \linebreak
    isready
} & \texttt{} \\ \hline

\texttt{} & \texttt{readyok} \\ \hline
\texttt{position startpos \linebreak
    go wtime 60000 btime 60000 winc 600 binc 600} & \texttt{} \\ \hline
\texttt{} & \texttt{
    info depth 1 nodes 21 pv g1f3\linebreak
    info depth 2 nodes 177 pv d2d4 g8f6\linebreak
    ... \linebreak
    bestmove d2d4} \\ \hline

\texttt{position startpos moves d2d4 d7d5\linebreak
go wtime 58000 btime 55000 winc 600 binc 600} & \texttt{} \\ \hline

\texttt{...} & \texttt{...} \\ \hline
\texttt{quit} & \texttt{} \\ \hline

\end{tabularx}
\end{table}



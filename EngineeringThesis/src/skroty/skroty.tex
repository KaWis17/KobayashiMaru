\pdfbookmark[0]{Skróty}{skroty.1}%
\chapter*{Skróty}
\label{sec:skroty}
\noindent\vspace{-\topsep-\partopsep-\parsep} % Je�li zaczyna si� od otoczenia description, to otoczenie to l�duje lekko ni�ej ni� wyl�dowa�by zwyk�y tekst, dlatego wstawiano przesuni�cie w pionie
\begin{description}[labelwidth=*]
  \item [UCI] (ang.\ \emph{Universal Chess Interface}) Uniwersalny Interfejs Szachowy
  \item [FEN] (ang.\ \emph{Forsyth–Edwards Notation}) Notacja Forsytha-Edwardsa
  \item [LAN] (ang.\ \emph{Long Algebraic Notation}) Pełna Algebraiczna Notacja Szachowa
  \item [FIDE] (fr.\ \emph{Fédération Internationale des Échecs}) Międzynarodowa Federacja Szachowa
  \item [Perft] (ang.\ \emph{Performance Test}) Test Wydajności
  \item [SPRT] (ang.\ \emph{Sequential Probability Ratio Test}) Sekwencyjnych Testów Probabilistycznych
  \item [MVV-LVA] (ang.\ \emph{Most Valuable Victim – Least Valuable Attacker}) Najbardziej Wartościowa Ofiara – Najmniej Wartościowy Atakujący
\end{description}



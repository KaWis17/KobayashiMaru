\begin{abstract}

    Celem pracy było opracowanie silnika szachowego w języku Java.
    Program ten~zaprojektowano tak, aby analizować i oceniać pozycję na szachownicy, a~następnie sugerować najlepszy ruch, uwzględniając jego strategiczne i~taktyczne aspekty.
    Interakcję z programem zaprojektowano dla wiersza poleceń, z~wykorzystaniem Uniwersalnego Interfejsu Szachowego.
    Umożliwiło to łatwą integrację z innymi aplikacjami szachowymi oraz pozwoliło na~przeprowadzanie symulacji i analiz działania silnika bez potrzeby aplikowania interfejsu graficznego.

    Niniejsza praca inżynierska składa się z trzech części, w których omówiono kolejne etapy pracy nad opracowaniem silnika szachowego.
    W pierwszej części przedstawiono podstawową wersję programu, która obejmuje generowanie możliwych ruchów zgodnie z zasadami gry w szachy, algorytm wyszukiwania optymalnego ruchu oraz implementację naiwnej heurystyki.
    W~części drugiej opisano ulepszenia algorytmów wyszukiwania i~oceny pozycji, mające na celu zwiększenie efektywności i precyzji silnika.
    Ostatnią część pracy poświęcono zagadnieniom związanym z testowaniem siły programu.
    Przeprowadzono analizę wydajności w odniesieniu do różnych jego wersji oraz innych silników, uwzględniając testy porównawcze oraz metodologię oceny skuteczności.

    Efektem prac jest silnik, którego ranking na platformie szachowej Lichess wynosi pomiędzy 1600 a 1700 ELO, co mieści go wśród 40\% najlepszych graczy.

\end{abstract}
\mykeywords

{
    \selectlanguage{english}
    \begin{abstract}

        The aim of this thesis is to develop a chess engine in Java.
        This program is designed to analyze and evaluate position on the chessboard and subsequently suggest the best move, considering its both strategic and tactical aspects.
        Interaction with the program is conducted via the command line, using the Universal Chess Interface.
        This allows for easy integration with other chess applications and~facilitates the simulation and analysis of the engine's performance without the need to create graphical interface.

        This engineering thesis consists of three parts, which discuss the~successive stages of developing the chess engine.
        The first part presents the basic version of the program, which includes generating possible moves according to the rules of chess, the algorithm for searching for the optimal move, and the implementation of~a~naive heuristic.
        The second part describes the improvements to the search and position evaluation algorithms, aiming to increase the efficiency and precision of~the~engine.
        The final part of the thesis is dedicated to issues related with testing the~engine's strength.
        An analysis of performance was conducted with respect to~various versions of the program and other engines, including comparative tests and methodology for~assessing effectiveness.

        The result of the work is an engine, whose chess ranking on Lichess platform can be estimated in the range between 1600 and 1700 ELO, which places it at the top 40\% of players.
    \end {abstract}
    \mykeywords}

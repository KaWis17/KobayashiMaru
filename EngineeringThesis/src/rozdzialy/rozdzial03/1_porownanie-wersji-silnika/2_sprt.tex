
\subsection{Sequential Probability Ratio Test}
\label{subsec:sequential-probability-ratio-test}
Testy porównawcze stosujące ustaloną liczbę rozgrywek, miały jednak szereg wad wpływających na wyniki testów.
Przede wszystkim, nie sposób było w tej metodzie ustalić, ile partii należy rozegrać, aby uzyskać wiarygodne statystycznie wyniki.
W przypadku dużej przewagi jednego z silnika wyniki można było uznać za autentyczne, jednak przy mniejszych różnicach powstawało ryzyko, że rozgrywki nie były reprezentatywne.

W celu zwiększenia wiarygodności wyników, zdecydowano się na zastosowanie Sekwencyjnych Testów Probabilistycznych (ang.\ \emph{Sequential Probability Ratio Test}, SPRT).
\subsubsection{Metodologia}
    Who knows? I dont.
\subsubsection{Przeprowadzenie testów}
    W celu przeprowadzenia SPRT zdecydowano się na ponowne zastosowanie Cutechess.
\subsubsection{Omówienie wyników}
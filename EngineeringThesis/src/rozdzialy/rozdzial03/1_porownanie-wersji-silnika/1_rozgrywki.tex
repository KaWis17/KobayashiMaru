
\subsection{Metodologia}
Najbardziej oczywistą, a zarazem najłatwiejszą do zaimplementowania metodą porównawczą jest rozegranie określonej liczby pojedynków pomiędzy oponentami, a następnie sprawdzenie współczynnika wygranych.
Do przeprowadzenia testów wykorzystano program komputerowy Cutechess, służący nie tylko jako szachowy interfejs graficzny, ale także jako zarządca turniejowy dla silników implementujących protokół UCI.

Pomiędzy parami silników rozegrano po 25 gier, każda trwająca 1 minutę na partię plus 0,6 sekundy na ruch.
Aby zmniejszyć powtarzalność partii, gry rozpoczynały się nie od pozycji startowej, ale od losowych otwarć szachowych z biblioteki. \cite{lichess-book}
Pozycje te były wyrównane, aby~uniknąć przewagi początkowej któregokolwiek z silników.

\newpage
\subsection{Wyniki}


\subsubsection{Tablica figur}
\begin{figure}[ht]
    \centering
    \includegraphics[width=1\linewidth]{rozdzialy/rozdzial03/1_porownanie-wersji-silnika/rysunki/wyniki-tablica}
    \caption{Wyniki rozgrywek z tablicą figur}
    \label{fig:wyniki-tablica}
\end{figure}
Silnik z zaimplementowaną tablicą figur uzyskał znaczną przewagę nad swoim rywalem.
Nie tylko nie przegrał ani jednej partii, ale wygrał aż 80\% z nich.
Świadczy to o zwiększeniu dokładności oceny heurystycznej pozycji, co przełożyło się na możliwość podejmowania lepszych decyzji w trakcie gry.


\subsubsection{Struktura pionów i bezpieczeństwo króla}
\begin{figure}[ht]
    \centering
    \includegraphics[width=1\linewidth]{rozdzialy/rozdzial03/1_porownanie-wersji-silnika/rysunki/wyniki-full-eval}
    \caption{Wyniki rozgrywek ze strukturą pionów i bezpieczeństwem króla}
    \label{fig:wyniki-full-eval}
\end{figure}
Poprzednią wersję implementującą tablicę figur porównano z wersją, która dodatkowo posiadała ulepszenia w postaci struktury pionów oraz oceny bezpieczeństwa króla.
Wyniki były bardziej zbliżone niż w poprzednim pojedynku, co sugeruje ich mniejszy wpływ na siłę silnika.
Niemniej, były one na tyle znaczące, aby zagwarantować wygraną na poziomie 48\%.
Z analizy pojedynczych partii wynikało, że silnik z ulepszeniami zyskiwał przewagę już w pierwszych ruchach gry, co dalej pozwalało mu na kontrolę nad planszą w późniejszych etapach.


\subsubsection{Alfa-Beta cięcie}
\begin{figure}[ht]
    \centering
    \includegraphics[width=1\linewidth]{rozdzialy/rozdzial03/1_porownanie-wersji-silnika/rysunki/wyniki-alfa-beta}
    \caption{Wyniki rozgrywek z alfa beta cięciem}
    \label{fig:wyniki-alfa-beta}
\end{figure}
Pierwsze sprawdzane ulepszenie co do algorytmów przeszukiwania drzewa gry przyniosło oczekiwane rezultaty.
Zmniejszenie liczby odwiedzonych węzłów pozwoliło na zwiększenie głębokości przeszukiwań, a co za tym idzie, pozwoliło silnikowi na lepsze ocenianie pozycji.
Wynik 64\% wygranych oraz 24\% remisów utwierdza w przekonaniu o poprawności i~skuteczności implementacji algorytmu alfa-beta.


\subsubsection{Ewaluacja stanów cichych}
\begin{figure}[ht]
    \centering
    \includegraphics[width=1\linewidth]{rozdzialy/rozdzial03/1_porownanie-wersji-silnika/rysunki/wyniki-stany-ciche}
    \caption{Wyniki rozgrywek z ewaluacją stanów cichych}
    \label{fig:wyniki-stany-ciche}
\end{figure}
Wykonanie testów dla tego ulepszenia było szczególne istotne, z tego względu, że dodatkowa ewaluacja stanów cichych ma także tendencje do zwiększenia liczby odwiedzonych węzłów drzewa.
Należało potwierdzić, że zysk związany z uniknięciem efektu horyzontu przewyższa koszty związane z dodatkowymi obliczeniami.
Jak widać na obrazku \ref{fig:wyniki-stany-ciche}, silnik z ulepszeniem wygrał 56\% i zremisował 32\% partii, co sugeruje, że dodatkowe obliczenia były warte podjęcia.


\subsubsection{Statyczne sortowanie ruchów}
\begin{figure}[ht]
    \centering
    \includegraphics[width=1\linewidth]{rozdzialy/rozdzial03/1_porownanie-wersji-silnika/rysunki/wyniki-sortowanie}
    \caption{Wyniki rozgrywek ze statycznym sortowaniem ruchów}
    \label{fig:wyniki-sortowanie}
\end{figure}
Wyniki pojedynku pomiędzy poprzednią wersją silnika a wersją z zaimplementowanym statycznym sortowaniem ruchów były zaskakująco korzystne.
Silnik z ulepszeniem wygrał 68\% partii, przy jednoczesnym braku jakiejkolwiek porażki.
Jeszcze raz potwierdza to znaczenie ułożenia odwiedzanych wierzchołków dla algorytmu alfa-beta.

\subsubsection{Tabela transpozycji}
\begin{figure}[ht]
    \centering
    \includegraphics[width=1\linewidth]{rozdzialy/rozdzial03/1_porownanie-wersji-silnika/rysunki/wyniki-transpozycje}
    \caption{Wyniki rozgrywek z tabelą transpozycji}
    \label{fig:wyniki-transpozycje}
\end{figure}
Otrzymane rezultaty dla ulepszenia związanego z tabelą transpozycji mogły wskazywać na błąd w implementacji.
Silnik, zamiast poprawić wynik swojego poprzednika, remisował większość pojedynków, nie wygrywając żadnego.
Co jeszcze bardziej zaskakujące, analiza poszczególnych rozgrywek wskazywała, że silnik z ulepszeniem w większości zremisowanych gier obejmował prowadzenie.
W opinii autora rezultat taki mógł być po części efektem nierozróżniania oceny tej samej pozycji, która wystąpiła trzykrotnie.
W efekcie silnik zapętlał się w pozycji z jego perspektywy korzystnej, prowadząc do remisu.
Poprawa implementacji wymagałaby stworzenia oddzielnego haszowania rozróżniającego te dwie pozycje.
Jako że tabela transpozycji oraz okno estymacji i tak nie mogły zostać wykorzystane jednocześnie, zdecydowano się na porzucenie tego ulepszenia.

\subsubsection{Okno estymacji}



\subsection{Wnioski}

Powyżej przedstawiono te z wyników, które miały najwyższy wpływ na siłę silnika, bądź przedstawiały ciekawe zależności.
Biblioteka otwarć, choć istotna, nie została porównana, gdyż rozgrywki rozpoczynały się od wyrównanych pozycji w grze środkowej, a więc biblioteka otwarć nie miała wpływu na wynik.

Tam, gdzie wynik był bardziej wyrównany, należałoby przeprowadzić testy na większej próbce badawczej, bądź wykonać testy korzystające z odmiennej, bardziej formalnej metodologii.
Powszechnie stosowaną techniką jest Sekwencyjny Test Probabilistyczny (ang.\~\emph{Sequential Probability Ratio Test}, SPRT), który pozwala na zwiększenie wiarygodności wyników, a także na zredukowanie do minimy liczby rozgrywek potrzebnych do uzyskania wyników. \cite{SPRT}
Program Cutechess posiada również możliwość przeprowadzenia testów SPRT, wykraczały one natomiast poza zakres niniejszej pracy.


%Największe różnice zaobserwowano w ulepszeniach:
%\begin{itemize}
%    \item \textbf{Tablica figur} – współczynnik wygranej dla tego ulepszenia sięgnął 90\%.
%    Wynikło to po części z tego, że silnik uzyskał o wiele więcej informacji co do pozycjonowania konkretnych figur na planszy.
%    Z drugiej strony mogło to wynikać z przewidywalności ruchów z uwagi na determinizm silnika.
%    Po zaimplementowaniu tablicy otwarć gry były bardziej losowe.
%    \item \textbf{Sortowanie ruchów} – lorem ipsum dolor sit amet, consectetur adipiscing elit.
%\end{itemize}
%
%%\subsection{Najmniej istotne ulepszenia}
%Najmniejsze różnice zaobserwowano natomiast w ulepszaniach:
%\begin{itemize}
%    \item \textbf{Ochrona króla} – wersja ulepszona, jak i poprzednia otrzymały po 50\% wygranych.
%    Przypuszczeniem autora jest, że zmiana heurystyki dotyczyła bardzo specyficznych sytuacji, które nie miały dużego wpływu na ogół rozgrywek.
%    \item \textbf{Biblioteka otwarć} – nie porównano wersji z biblioteką otwarć, gdyż rozgrywki rozpoczynają się od wyrównanych pozycji w grze środkowej, a więc biblioteka otwarć nie miałaby wpływu na wynik.
%\end{itemize}
%
%\begin{figure}[ht]
%    \centering
%    \includegraphics[width=1\linewidth]{rozdzialy/rozdzial03/1_porownanie-wersji-silnika/rysunki/gry-wyniki}
%    \caption{Wyniki rozgrywek pomiędzy wersjami silnika}
%    \label{fig:wyniki-wersje}
%\end{figure}
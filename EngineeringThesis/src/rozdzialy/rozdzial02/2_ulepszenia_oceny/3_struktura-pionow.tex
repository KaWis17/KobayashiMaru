\subsection{Struktura pionów}
\label{subsec:struktura-pionow}

\subsubsection{Opis zagadnienia}
Po implementacji ulepszenia \ref{subsec:tablice-figur} silnik uzyskał zdolność oceny pozycyjnej.
Problemem było natomiast, że bierki oceniane były pojedynczo, to jest bez uwzględnienia ich wzajemnych relacji, oraz relacji w stosunku do bierek przeciwnika.

Termin \enquote{struktura pionów} jest szeroko znany w literaturze szachowej \cite*{stuktura-pionow} i odnosi się do technik mających na celu skoordynowaną obronę własnej części szachownicy oraz przeprowadzania ataków z wykorzystaniem pionów.
Skuteczne pozycjonowanie pionków na planszy może prowadzić do uzyskania przewagi w grze.

\subsubsection{Implementacja}
Zasady rozumiane przez graczy szachowych w sposób intuicyjny należało przekształcić na zbiór reguł możliwych do zrozumienia przez program, a prowadzących do skutecznej oceny pozycji.
\begin{itemize}
    \item Pionki, które posiadają z tyłu po swojej prawej lub lewej stronie innego pionka, są uważane za dobrze chronione.
    \item Pionki nieposiadające za sobą innych pionków na sąsiednich kolumnach są traktowane jako izolowane i trudne do obrony.
    \item Dwa pionki znajdujące się w jednej linii pionowej (tzw. \enquote{zdublowane}) są uważane za~małowartościowe, gdyż odsłaniają inne linie na potencjalne ataki.
    \item Pionki, dla których istnieje możliwość promocji, spowodowana brakiem pionów przeciwnika na sąsiednich kolumnach, są uznawane za bardzo wartościowe.
\end{itemize}

Powyższe techniki zostały w dużej mierze zaimplementowane z wykorzystaniem uprzednio zainicjowanych tablic bitowych bierek i operacjach na nich.
Odbyło się to w sposób analogiczny do generowania dostępnych ruchów.

\subsubsection{Rezultat}
Rezultatem stał się silnik, który w sposób bardziej złożony oraz koherentny był w stanie ocenić swoją sytuację na planszy.
Istniało natomiast ryzyko, że czas konieczny na przeprowadzanie obliczeń znacząco przewyższy oferowane korzyści.
%Nie sposób było to sprawdzić w trakcie zwykłej gry przeciw silnikowi.
%Rzeczywiste efekty przedstawiono w rozdziale \ref{ch: ocena-sily-silnika}.
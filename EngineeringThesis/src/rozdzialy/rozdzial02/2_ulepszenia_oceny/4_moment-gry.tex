\subsection{Moment gry}
\label{subsec:moment-gry}

\subsubsection{Opis zagadnienia}
Szachy to gra dynamiczna, w której techniki stosowane przez szachistów w dużym stopniu zależą od pozycji na planszy.
Jednym z podstawowych wskaźników wpływających na obierane strategie jest moment gry.
Partia szachowa dzieli się na trzy etapy: \textbf{otwarcie} – najczęściej trwające do~10~ruchów, rozgrywane z zapamiętanej teorii szachowej, \textbg{grę środkową} – najdłuższy etap, w którym obie strony walczą o kontrolę nad planszą, oraz \textbf{grę końcową} – etap, w którym ilość bierek po obu stronach jest niska, a gracze próbują znaleźć techniki zamatowania przeciwnika.

\subsubsection{Implementacja}
Pierwszy z nich zaimplementowany został dzięki Bibliotece Otwarć \ref{subsec:biblioteka-otwarc}.
W celu odróżnienia gry środkowej od końcowej zastosowano kryterium, mówiące o pozostaniu po każdej stronie maksymalnie jednej figury wysokiej (hetmana, wieży) bądź dwóch figur niskich (gońca, skoczka).
W przypadku spełnienia tego warunku silnik przechodzi do fazy końcowej, cechującej się odmienną oceną.

Między innymi zmienia się tablica dla figury króla.
W fazie środkowej król zazwyczaj unikał walki bezpiecznie schroniony po roszadzie w rogu planszy.
Natomiast w fazie końcowej, z~uwagi na mniejszą ilość materiału zyskuje on na znaczeniu.
Powinien przesunąć się do centrum planszy, pomagając w ochronie pionków oraz znajdowaniu matów na przeciwniku.
Co więcej, na znaczeniu zyskały same pionki, gdzie ich promocja na hetmana może przesądzić o losie partii.

\subsubsection{Rezultat}
Zastosowane ulepszenie przyniosło zakładane rezultaty.
W połączeniu ze zwiększeniem głębokości wyszukiwania, wynikającej z mniejszej liczby bierek, a co za tym idzie niższym współczynnikiem rozgałęzienia drzewa, doprowadziło do efektywnej gry końcowej.
Wyszukiwanie ruchów było na tyle dobre, że podjęto decyzję o zaniechaniu implementacji siatek matowych, gdyż silnik radził sobie bardzo dobrze ze znajdowaniem rozwiązań.
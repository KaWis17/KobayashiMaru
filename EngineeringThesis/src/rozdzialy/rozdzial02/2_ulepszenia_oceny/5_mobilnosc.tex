\subsection{Mobilność}
\label{subsec:mobilnosc}

\subsubsection{Opis zagadnienia}
Mobilność bierek mówi o ilości posunięć dostępnych dla danej figury.
Celem jest zwiększenie liczby dostępnych ruchów, a co za tym idzie, kontroli większej części planszy, zmuszając przeciwnika do wycofania.
Tablice figur pozwoliły na wskazanie typowych pól, które są korzystne z punktu widzenia hetmana, wierzy czy gońca.
Istnieją jednak sytuacje, w których warto ustawić się na polu obiektywnie mniej korzystnym, natomiast umożliwiającym kontrolę otwartej linii bądź przekątnej.

\subsubsection{Implementacja i rezultat}
Do implementacji zastosowano generator ruchów, który wskazywał ilość dostępnych posunięć dla konkretnej figury.
Wartość ta, przekazywana do funkcji heurystycznej i mnożona przez odpowiedni współczynnik, pozwalała na ocenę mobilności.
W wyniku implementacji powstał silnik, który lepiej oceniał poszczególne dostępne ruchy.
Niestety, zwiększenie liczby koniecznych obliczeń zdecydowanie obniżyło wydajność.
Efektywna implementacja tego ulepszenia, choć możliwa, wymagałaby zastosowania bardziej zaawansowanych technik generowania ruchów.
Zdecydowano się na zrezygnowanie z tego ulepszenia na rzecz innych, bardziej obiecujących pod względem efektywności.
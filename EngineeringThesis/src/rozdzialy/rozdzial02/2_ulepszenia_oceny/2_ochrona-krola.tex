\subsection{Ochrona króla}
\label{subsec:ochrona-krola}

W celu próby zapobieżenia sytuacjom, w których król jest zagrożony, zaimplementowano mechanizm, który sprawdza wszystkie bierki w najbliższej okolicy króla.
Pionki i figury tego samego koloru są traktowane korzystnie, przeciwnego natomiast bardzo niekorzystnie.
Co~więcej, istotny jest typ figury, która zagrożenie stanowi.
Miało to w efekcie ograniczyć ryzyko zakładania mata, szczególnie na środkowym etapie gry.
Jest to ulepszenie, którego skuteczność jest trudna do zweryfikowania w czasie gry, z uwagi na fakt, że jego efekt jest widoczny dopiero w sytuacji zagrożenia króla.


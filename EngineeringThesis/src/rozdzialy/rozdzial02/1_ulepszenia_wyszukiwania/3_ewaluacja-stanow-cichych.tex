\usepackage{biblatex}\subsection{Ewaluacja cichych stanów}
\label{subsec:ewaluacja-cichych-stanow}

\subsubsection{Opis zagadnienia}
Silnik znacznie przyspieszył znajdowanie ruchów dla danej głębokości.
Zdarzały się jednak sytuacje, że zwracane wyniki znacznie odbiegały od optymalnych.
Dla przykładu mogła zdarzyć się sytuacja, że liściem w drzewie było bicie hetmanem skoczka.
Silnik zwracał wynik heurystyki jako bardzo wysoki i wykonywał powyższy ruch.
Program nie brał jednak pod uwagę faktu, że~na~głębokości o jednej większej, hetman ten mógłby zostać zbity przez wrogiego pionka.

Taką sytuację nazywa się efektem horyzontu.
Rozwiązaniem wykonywania nieoptymalnych ruchów, było wykonanie dodatkowego wyszukiwania po osiągnięciu głębokości końcowej.
Z liścia drzewa, które poprzednio zostało oceniane heurystycznie, wyprowadzono kolejne wyszukiwanie, w którym generowane ruchy są tylko biciami.
Pozwoliło to na ocenę tak zwanych stanów cichych, czyli pozycji, gdzie nie ma żadnego dostępnego ruchu, który prowadziłby do~bicia.

\subsubsection{Implementacja}

Algorytm rozwiązujący efekt horyzontu w literaturze nosi nazwę Quiescence Search.

\begin{lstlisting}[
    language=Java,
    style=JavaStyle,
    caption=Implementacja algorytmu Quiescence Search,
    label=lst:trzeci]
    public search(depth, alpha, beta) {
        standPat = evaluator.evaluate();
        if (depth == 0) return standPat;

        if (standPat >= beta) return beta;
        if (alpha < standPat) alpha = standPat;

        moves = generator.generateCaptureMoves();
        for(Move move : moves) {
            board.makeMove(move);
            score = -search(depth-1, -beta, -alpha);
            board.unmakeMove();

            if(score >= beta) return beta;
            if(score > alpha) alpha = score;
        }
        return alpha;
    }
\end{lstlisting}

Niektóre wersje tego algorytmu, oprócz ruchów prowadzących do bicia, korzystają także z~roszad, czy~podwójnych ruchów piona, z uwagi na ich szczególny charakter.

\subsubsection{Rezultat}

W efekcie implementacji algorytmu osiągnięto silnik, który jest odporny na efekt horyzontu.
Z reguły, dodatkowe wyszukiwanie prowadzi do odwiedzenia większej liczby węzłów niż w przypadku zwykłego wyszukiwania.
Co warte zaznaczenia natomiast, w niektórych przypadkach, chociażby wskazanych w dodatku \ref{ch:wyniki-perft}, algorytm Quiescence Search w połączeniu z alfa beta cięciem skutkował zmniejszeniem liczby odwiedzonych węzłów.
Ostatecznie, czy korzystniejszym jest wyszukiwanie dla większych głębokości, czy korzystanie wyłącznie ze~stanów cichych zależy od konkretnej sytuacji na planszy.
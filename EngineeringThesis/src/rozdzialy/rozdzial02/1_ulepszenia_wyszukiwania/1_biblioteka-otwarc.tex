\subsection{Biblioteka otwarć}
\label{subsec:biblioteka-otwarc}

\subsubsection{Opis zagadnienia}
Głównym problemem, który należało rozwiązać, były otwarcia partii rozgrywanych przez silnik.
Gry szachowe zawsze rozpoczynają się z tego samego położenia, co oznacza, że~system musi znaleźć się w pozycjach bezpośrednio wynikających z pierwszych posunięć.
Program poświęcał na początkowe obliczania zbyt wiele czasu, który lepiej byłoby spożytkować w środkowym etapie rozgrywki.

Profesjonalni gracze, korzystając z dorobku i doświadczenia wielu pokoleń szachistów, zapamiętują pierwsze posunięcia do wykonania.
Tym samym oszczędzają czas, który pozostaje na bardziej złożone pozycje w środkowej fazie gry.

\subsubsection{Implementacja}
W celu zaimplementowania biblioteki otwarć, dostępne posunięcia czytane są z pliku, a~następnie zapisywane przy inicjalizacji silnika do HashMapy \texttt{<String, []Moves>}.
Podczas działania silnika, FEN aktualnej pozycji porównywalny jest z dostępnymi kluczami, a ruch losowo wybierany spośród dostępnych w tablicy.
W przypadku nieznalezienia klucza program przechodzi do gry środkowej z wykorzystaniem algorytmu negaMax.

W literaturze znanych jest wiele formatów zapisu biblioteki otwarć, np.\ EPD, PGN czy~Bin-format.
System wykorzystuje bibliotekę wygenerowaną przez Sebastiana Lague i~dostępną na licencji MIT w~implementacji jego silnika~\cite*{opening-library}.
\subsubsection{Rezultat}
Efektem prac stał się silnik, który początkowe posunięcia wykonuje za pomocą podpiętej biblioteki otwarć.
Dzięki temu pierwsze ruchy wykonywane są bardzo szybko, pozostawiając cenny czas obliczeniowy na dalszą grę i tym samym zwiększając precyzję.

We wcześniejszej wersji silnik był deterministyczny, to jest, dla konkretnej głębokości i pozycji, algorytm zwracał ten sam najlepszy ruch.
Dodatkowym atutem stała się jego nieprzewidywalność.
Początkowe posunięcia są zrandomizowane, co pozwala na jego testowanie w większej liczbie wariantów, a także umila rozgrywkę użytkownikowi.

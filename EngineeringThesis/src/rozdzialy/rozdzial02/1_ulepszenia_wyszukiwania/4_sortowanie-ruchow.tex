\subsection{Statyczne sortowanie ruchów}
\label{subsec:sortowanie-ruchow}

\subsubsection{Opis zagadnienia}
W czasie testów zauważono, że podczas przeszukiwania drzewa nie następuje taka ilość alfa-beta cięć, która pozwalałaby na znaczne przyspieszenie obliczeń.
Rozpoczynanie przeszukiwania węzłów gry od najlepszych posunięć pozwoliłoby na osiągnięcie teoretycznego limitu wskazanego w tabeli \ref{tab:alfa-beat-limit}.
W żaden sposób nie ma jednak możliwości wskazania, który ruch prowadzi do najlepszej pozycji, przed wykonaniem tych ruchów.
Istnieją jednak posunięcia, które można rozważyć w pierwszej kolejności, ze względu na ich szczególny charakter, a~tym~samym przyspieszyć cięcie.
Jeśli sortowanie jest niezależne od wcześniejszych obliczeń, to~jest to sortowanie statyczne.

\subsubsection{Implementacja}
Sortowanie wykonano dzięki implementacji interfejsu \texttt{Comparator} w klasie \texttt{Move}.
W celu porównania pomiędzy sobą dwóch dowolnych ruchów wykorzystano następujące założenia:
\begin{itemize}
    \item Ruch, który umożliwia bicie, jest prawdopodobnie lepszy od ruchów, które nie umożliwiają bicia.
    \item Wszystkie ruchy specjalne, takie jak roszady, podwójne ruchy pionem czy promocje, są lepsze od ruchów zwykłych.
    \item Promocje do wieży i gońca należy rozważyć na końcu, gdyż są mniej wartościowe od promocji do hetmana, jednocześnie dając dostęp do tych samych ruchów.
    \item Zastosowano technikę Najbardziej wartościowa ofiara – Najmniej wartościowa atakujący (ang.~\emph{Most Valuable Victim – Least Valuable Attacker}, w~skrócie MVV-LVA). Sortuje ona~bicia zakładając, że korzystniejsze są posunięcia, w których różnica pomiędzy wartością reprezentowaną przez bierkę bijącą a bierką zbitą, jest jak największa.
\end{itemize}

\subsubsection{Rezultat}
Dzięki zastosowaniu sortowania statycznego, udało się zwiększyć ilość alfa-beta cięć, co~spowodowało możliwości przeszukiwania drzew gry do większych głębokości.
Szczególnie zauważalne było to ulepszenie w procesie wyszukiwania stanów cichych, gdzie każdy z~dostępnych ruchów można było zakwalifikować za pomocą MVV–LVA.
Dodatek \ref{ch:wyniki-perft} zawiera przykładowe pozycje, w których porównano wyniki perft dla wersji klasycznej, z alfa-beta cięciem, oraz z sortowaniem ruchów.
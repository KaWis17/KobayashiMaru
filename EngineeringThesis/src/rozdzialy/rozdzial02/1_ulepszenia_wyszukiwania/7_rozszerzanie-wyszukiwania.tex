\subsection{Rozszerzanie wyszukiwania}
\label{subsec:rozszerzanie-wyszukiwania}

\subsubsection{Opis zagadnienia}
Rozszerzenie wyszukiwania to technika, która zakłada, że niektóre z dostępnych posunięć wymagają dodatkowego zbadania.
W przypadku, gdy taki ruch nastąpi, algorytm przeszukiwania przedłuży przeszukiwanie poddrzewa gry o jeden poziom.
W odróżnieniu od~Quiescence Search rozszerzenie wyszukiwania przedłuża się na wszystkie z dostępnych ruchów z pozycji, a nie tylko na te, które prowadzą do bicia.

\subsubsection{Implementacja}
Zdecydowano się na implementację dwóch rodzajów przedłużeń:
\begin{itemize}
    \item Jeśli ruch jest atakiem powodującym wystąpienie szacha, to przeszukiwanie zostaje przedłużone o jeden poziom.
    \item W przypadku, gdy istnieje jedynie jedno dostępne posunięcie, również przeszukiwanie zostaje przedłużone o jeden poziom.
\end{itemize}
W celu uniknięcia zbytniego rozprzestrzeniania się grafu, przedłużenie wyszukiwania może nastąpić w danym poddrzewie gry tylko raz.
\subsubsection{Rezultat}
W toku gry nie zaobserwowano znacznego wzrostu bądź spadku siły silnika.
Efektywność rozwiązania mogła zostać zweryfikowane w dalszym rozdziale.
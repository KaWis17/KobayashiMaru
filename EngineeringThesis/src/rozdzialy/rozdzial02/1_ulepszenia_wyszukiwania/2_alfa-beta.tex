\subsection{Alfa-Beta cięcie}
\label{subsec:alfa-beta-ciecie}
\subsubsection{Opis zagadnienia}
Poprawne wykonanie algorytmu negaMax nie gwarantuje jego skuteczności.
Wraz z~pogłębianiem wyszukiwania rośnie również liczba wierzchołków, które należy odwiedzić.
Uśredniając, liczba dzieci, to jest pozycji wynikających z wykonania ruchu, dla danego rodzica w szachach wynosi $35$~\cite*{branching-factor}.
Własność ta nazywa się współczynnikiem rozgałęzienia (ang.~\textit{branching~factor}).

Do minimalizacji liczby odwiedzonych węzłów wykorzystuje się alfa-beta cięcia.
Jest to przykład metody podziału i ograniczeń, a skuteczność w dużym stopniu zależy~od kolejności przeszukiwanych ruchów.
Zakładając ułożenie posunięć od najlepszego, algorytm wykona najwięcej cięć, co w efekcie skutkować będzie zmniejszeniem liczby odwiedzonych wierzchołków dla głębokości według wzoru z tabeli:

\begin{table}[htb] \small
\centering
\caption{Teoretyczny limit alfa-beta cięcia dla współczynnika rozgałęzienia $b_f = 35$}
\label{tab:alfa-beat-limit}
\renewcommand{\arraystretch}{1.5}
\begin{tabular}{|c|c|c|}\hline
Głębokość $n$ & ${b_{f}}^{n}$ & $b_{f}^{\lceil \frac{n}{2} \rceil} + b_{f}^{\lfloor \frac{n}{2} \rfloor} - 1$\\ \hline\hline

$1$ & $35$ & $35$\\ \hline
$2$ & $1225$ & $69$\\ \hline
$3$ & $42 875$ & $1259$\\ \hline
$\vdots$ & $\vdots$ & $\vdots$\\ \hline
$10$ & $2,76 * 10^{15}$ & $1,05 * 10^{8}$\\ \hline

\end{tabular}
\end{table}

Posortowanie ruchów od najgorszych, spowoduje brak możliwych cięć, a tym samym wynik będzie identyczny, jak przy algorytmie bez implementacji alfa-beta cięcia.


\subsubsection{Implementacja}

Algorytm alfa-beta cięcia jest edycją kodu funkcji negaMax \ref{lst:drugi}.
$\alpha$ oraz $\beta$ zostają dodane jako argumenty.
W sytuacji, gdy \texttt{score} $>$ \texttt{bestMoveValue} oraz $\alpha >$ \texttt{bestMoveValue}, $\alpha$~zostaje zaktualizowana na \texttt{bestMoveValue}.
Jeśli \texttt{score} $>=\beta$ następuje odcięcie i zwracana jest wartość \texttt{bestMoveValue}.
W wywołaniach funkcji na większej głębokości $\alpha = -\beta$ oraz $\beta = -\alpha$.

\subsubsection{Rezultat}

Implementacja powyższego rozwiązania pozwoliła na zmniejszenie liczby odwiedzonych wierzchołków podczas przeszukiwania, a tym samym pozwoliła iteratywnemu pogłębianiu na~osiągnięcie lepszych rezultatów.
Realna wartość tego ulepszenia jest różna w zależności od~aktualnej pozycji na szachownicy.
\subsection{Okno estymacji}
\label{subsec:okno-estymacji}

\subsubsection{Opis zagadnienia}
Poprzednie wersje silnika rozpoczynając przeszukiwanie węzłów gry, dla każdego poziomu głębokości, zakładały wartości alfa i beta jako odpowiednio $-\infty$ i $+\infty$.
W praktyce można jednak przypuszczać, że wartości zwrócone dla większych głębokości, nie będą w bardzo dużym stopniu odbiegać, od wartości zwróconych dla głębokości mniejszych.
Rozpoczynając wyszukiwanie algorytmem miniMax z wartościami $alpha = sc_{n-1} - \delta$ oraz $beta = sc_{n-1} + \delta$, gdzie $sc_{n-1}$ to wartość zwrócona dla poprzedniej głębokości, a $\delta$ to okno estymacji, istnieje szansa wystąpienia większej ilości alfa-cięć oraz beta-cięć.
Pomimo oczywistych korzyści wynikających z zastosowania powyższej techniki, istnieje także ryzyko, że w przypadku, gdy wartość wykracza poza okno estymacji, konieczne jest rozpoczęcie ponownego wykonania algorytm wyszukiwania.
Ponowne uruchomienie odbywa się z szerszym oknem estymacji.

\subsubsection{Implementacja}
Kluczowe w implementacji algorytm było odpowiednie ustalenie wartości $\delta$.
Zbyt wąskie okno skutkowałoby zbyt częstymi ponownymi przeszukiwaniami drzewa.
Zbyt szerokie natomiast, prowadziłoby do braku efektów zastosowanego usprawnienia.
Większość silników ustawia wartość $\delta$ na $1/3$ wartości pionka. \cite*{duch}
Taką też wartość zastosowano w programie.
W przypadku wyjścia poza zakres, okno estymacji poszerzone zostaje do $-\infty$ i $+\infty$, co w efekcie gwarantuje poprawne działanie.

\subsubsection{Rezultat}
\begin{center}
    \textcolor{red}{\textbf{JESZCZE NIE ZAIMPLEMENTOWANE}}
\end{center}

\subsection{Tabela transpozycji}
\label{subsec:tabela-transpozycji}

\subsubsection{Opis zagadnienia}
Silnik szachowy wielokrotnie znajduje się w tej samej pozycji planszy, musząc wyliczać wartości heurystyczne od nowa.
Dzieje się tak, zarówno w przypadku dotarcia do pozycji w wyniku odmiennych sekwencji ruchów, jak i w wyniku wykonywania kolejnych ruchów.
Rozwiązaniem tego problemu jest zastosowanie tabeli transpozycji, która przechowuje wyniki obliczeń dla pozycji, mając jednocześnie na uwadze głębokość, dla której wynik został obliczony.
W sytuacji, gdy silnik ponownie napotka na tę samą pozycję, w pierwszej kolejności sprawdzi, czy w tabeli nie znajduje się już obliczony wynik dla tej samej, bądź większej głębokości.

\subsubsection{Implementacja}
Do implementacji tabeli transpozycji zastosowano strukturę \texttt{LinkedHashMap}, dla której kluczami są wartości \texttt{Zobrist Hasz} planszy, a wartościami struktura zawierająca wynik obliczeń dla danej pozycji oraz głębokość.
W trakcie działania programu silnik przechodzi przez miliony możliwych pozycji, natomiast możliwości przechowywania wyników są ograniczone.
W przypadku przekroczenia z góry ustalonego limitu wpisów usuwane są te wartości, które zostały użyte najdawniej.
Pozwala to na pozbycie się poddrzew, w których silnik nie będzie miał okazji się ponownie znaleźć, przy jednoczesnym zachowaniu prostoty implementacji.

\subsubsection{Rezultat}
Po zastosowaniu ulepszenia zaobserwowano zmniejszenie liczby odwiedzonych węzłów w trakcie przeszukiwania drzewa gry.
Było to szczególnie widoczne przy mniejszych głębokościach, gdzie w dużej mierze pozycje zostały już wyliczone w trakcie poprzednich posunięć.

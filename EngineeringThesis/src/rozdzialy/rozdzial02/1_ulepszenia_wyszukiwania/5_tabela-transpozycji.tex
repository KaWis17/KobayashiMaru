\subsection{Tabela transpozycji}
\label{subsec:tabela-transpozycji}

\subsubsection{Opis zagadnienia}
Silnik szachowy wielokrotnie znajduje się w tej samej pozycji planszy, musząc wyliczać wartości heurystyczne od nowa.
Dzieje się tak, zarówno w przypadku dotarcia do pozycji w wyniku odmiennych sekwencji ruchów, jak i w wyniku wykonywania kolejnych ruchów.
Rozwiązaniem tego problemu jest zastosowanie tabeli transpozycji, która przechowuje wyniki obliczeń dla pozycji, mając jednocześnie na uwadze głębokość, dla której wynik został obliczony.
W sytuacji, gdy silnik ponownie napotka na tę samą pozycję, w pierwszej kolejności sprawdzi, czy w tabeli nie znajduje się wynik oceny heurystycznej dla tej samej, bądź większej głębokości.

\subsubsection{Implementacja i rezultat}
Do implementacji zastosowano strukturę \texttt{HashMap}, gdzie kluczem jest wartość \texttt{Zobrist Hasz} planszy, a wartością wynik obliczeń dla danej pozycji.
Po zastosowaniu ulepszenia zaobserwowano znaczący wzrost wydajności silnika dla mniejszych głębokości przeszukiwania drzewa gry.
Wynikało to z faktu napotykania tych samych pozycji w trakcie przeszukiwania poprzedniego posunięcia.
Nie przełożyło się to jednak na ogólny poziom gry silnika.
Program komputerowy zaczął docierać do głębokości, w których nie było już możliwości ukończenia całego poziomu w wyznaczonym czasie, nawet pomimo znajdowania par klucz-wartość dla niższych głębokości.

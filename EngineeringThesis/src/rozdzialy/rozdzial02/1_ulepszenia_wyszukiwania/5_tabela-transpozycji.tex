\subsection{Tabela transpozycji \colorbox{red}{TODO}}
\label{subsec:tabela-transpozycji}

{
    \color{red}
    \large Pozycje już policzone sa haszowane Zobrist hashing oraz zapisywane.
    Gdy ponownie (na danym poziomie!?!) natrafimy na ten sam hash, to zwracamy wartość, nie przeszukując niżej drzewa.
    (Czy można tego użyć przy move ordering także?!?)
}
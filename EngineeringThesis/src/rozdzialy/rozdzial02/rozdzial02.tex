\chapter{Ulepszenia dla silnika szachowego}
\label{ch:implementacja-silnika-szachowego}


\section{Ulepszenia dla wyszukiwania}
\label{sec:ulepszenia-dla-wyszukiwania}

\subsection{Biblioteka otwarć \colorbox{red}{TODO}}
\label{subsec:biblioteka-otwarc}

{
    \color{red}
    \large Biblioteka otwarć jest parsowana z pliku na hashmape (FEN, możliwe ruchy).
    Losowo wybierany ruch spośród możliwych.
    Pozwala na randomizację posunięć szczególnie na początku.

}

\subsection{Alfa-Beta cięcie \colorbox{yellow}{Implemented}}
\label{subsec:alfa-beta-ciecie}

{
    \color{red}
    \large Zaimplementowane, pozostało wytłumaczyć o co cmon.
Powołać się na Papadimitru
}


\subsection{Ewaluacja cichych stanów \colorbox{red}{TODO}}
\label{subsec:ewaluacja-cichych-stanow}

{
    \color{red}
    \large Po zakończeniu zwykłego min-max należy doprowadzić do stanu \("\)cichego\("\) to znaczy takiego, gdzie nie ma żadnych dostępnych bić ani roszad.
Inaczej może to zaburzyć poprawną interpretację pozycji przez heurystykę.
}


\subsection{Sortowanie ruchów \colorbox{red}{TODO}}
\label{subsec:sortowanie-ruchow}

{
    \color{red}
    \large Sortujemy ruchy zaczynając od roszad i bić w celu rozważenia ich na początku.
    Umożliwi to szybsze działanie alfa-beta cięcia i przez to rozważanie mniejszego drzewa decyzyjnego.
}

\subsection{Tabela transpozycji \colorbox{red}{TODO}}
\label{subsec:tabela-transpozycji}

{
    \color{red}
    \large Pozycje już policzone sa haszowane Zobrist hashing oraz zapisywane.
    Gdy ponownie (na danym poziomie!?!) natrafimy na ten sam hash, to zwracamy wartość, nie przeszukując niżej drzewa.
    (Czy można tego użyć przy move ordering także?!?)
}


\subsection{Okno estymacji \colorbox{red}{TODO}}
\label{subsec:okno-estymacji}

{
    \color{red}
    \large Z tego co rozumiem, zakłada to, że znajdziemy minimum ruch o danej jakości i przez to odcinamy alfa-beta szybciej.
Jest jednak ryzyko, że takiego nie znajdziemy i będziemy musieli szukać od zera w większym oknie
}


\subsection{Rozszerzanie wyszukiwania \colorbox{red}{TODO}}
\label{subsec:rozszerzanie-wyszukiwania}

{
    \color{red}
    \large Wydłużenie o maksymalnie dwa przeszukiwania na danej głębokości o ile jest to ruch szczególny.
}


\section{Ulepszenia dla oceny heurystycznej}
\label{sec:ulepszenia-dla-oceny-heurystycznej}

{
    \color{red}
    \large Gdzieś widziałem, że do elo 1500 o wiele istotniejsze są ulepszenia co do heurystyki niż wyszukiwania.
Trzeba znaleźć linka do źródła
}


\subsection{Tablice figur \colorbox{yellow}{Implemented}}
\label{subsec:tablice-figur}

{
    \color{red}
    \large Tablice dla każdej ze stron i każdej z figur w celu przypisania punktacji.
}

\subsection{Ochrona króla \colorbox{red}{TODO}}
\label{subsec:ochrona-krola}

{
    \color{red}
    \large Sprawdzanie, czy król jest chroniony, na przykład przez piony
}

\subsection{Struktura pionów \colorbox{red}{TODO}}
\label{subsec:struktura-pionow}

{
    \color{red}
    \large Czy piony są w rządku, czy chronią siebie wzajemnie.
Najłatwiej to chyba przez bit-boardy sprawdzać
}


\subsection{Moment gry \colorbox{red}{TODO}}
\label{subsec:moment-gry}

{
    \color{red}
    \large Początek - środek - koniec.
    Różne wartości, szczególnie dla króla w zależności od momentu gry.
}


\subsection{Mobilność \colorbox{red}{TODO}}
\label{subsec:mobilnosc}

{
    \color{red}
    \large Możliwość ruchów konkretnych figur, szczególnie gońców i skoczków.
}

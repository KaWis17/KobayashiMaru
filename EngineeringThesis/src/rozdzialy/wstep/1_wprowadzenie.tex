\section{Wprowadzenie}
\label{sec:wprowadzenie}

Początki szachów, wywodzących się z Indii, datuje się na VI wiek.
Od tamtych czasów gra królewska znacząco ewoluowała, przemierzając Persję i świat arabski, aż w końcu dotarła do Europy, gdzie stała się cenioną rozgrywką dworską.
Dzięki prostocie zasad, a zarazem złożoności strategii, gra ta szybko zyskały zainteresowanie wielu, zarówno profesjonalnych graczy biorących udział w turniejach, jak i amatorów poszukujących intelektualnych wyzwań.
Obecnie szachy stały się najpopularniejszą grą planszową w historii ludzkości.


W swojej 1500-letniej historii zmieniały się nie tylko oficjalne zasady gry, jak na przykład wprowadzenie roszady, czy ruchu en-passant, ale także stosowane techniki i strategie, mające zagwarantować zwycięstwo.
Szczególne znaczenie, dla rozwoju tych strategii, miał rozwój maszyn liczących w XX wieku, co otworzyło możliwość zautomatyzowania procesu analizy partii szachowych.


Za pioniera w tej dziedzinie uważany jest amerykański matematyk Claude Shannon, który w roku 1950 opublikował pracę o teoretycznych aspektach programowania silników szachowych, opartych o ocenę heurystyczną oraz algorytm min-max.
Istotny wkład w rozwój szachowej sztucznej inteligencji miał także ojciec informatyki, Alan Turing, który rok później zaprojektował pierwszy program komputerowy, w pełni zdolny do gry w szachy.
Ograniczenia techniczne tamtych czasów nie pozwoliły jednak przetestowania programu na maszynie.
Rozegrano parę partii szachowych, w których każdy ruch był obliczany analogowo.


Najstarszy program uruchomiony na komputerze, który pozwalał na przeprowadzenie pełnej rozgrywki, powstał w 1958 roku.
Od tamtego momentu wiele silników szachowych biło rekordy swoich poprzedników.
Aż do pamiętnej zimy 1997 roku, kiedy to silnik szachowy DeepBlue wygrał pojedynek $3\frac{1}{2} - 2\frac{1}{2}$ z ówczesnym mistrzem świata, Garrym Kasparovem.


Po tym wydarzeniu wkroczyliśmy w erę super silników.
Wykorzystanie komputerów stanowi nieodłączny element analizy partii szachowych.
Maszyny zadziwiają swoją precyzją, znajdując ruchy, których nie potrafią dostrzec najlepsi współcześni gracze.
Szachy stały się nie tylko areną dla ludzkiego intelektu, ale także polem testowym zaawansowanych technologii.
Zastosowanie najnowocześniejszych rozwiązań, takich jak sieci neuronowe, zrewolucjonizowało sposób, w jaki rozumiemy tą gre oraz pokazało, jak wiele jeszcze można w tej dziedzinie osiągnąć.

Era informatyki jest nie tylko nowym rozdziałem w historii szachów, ale także dowodem na to, jak daleko może zaprowadzić nas współpraca człowieka z maszyną.

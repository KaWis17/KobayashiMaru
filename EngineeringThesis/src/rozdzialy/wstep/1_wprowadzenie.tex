\section{Wprowadzenie}
\label{sec:wprowadzenie}

Szachy, których początki sięgają VI wieku w Indiach, stały się najpopularniejszą grą planszową w historii.
Dzięki swojej złożoności oraz unikalności każdej rozgrywki, ta królewska gra zdobyła zainteresowanie wielu, zarówno amatorów poszukujących intelektualnych wyzwań, jak i graczy profesjonalnych.
Na przestrzeni lat zmianom ulegały nie tylko oficjalne zasady gry, ale także styl i technika gry reprezentowane przez mistrzów szachowych.
Wraz z rozwojem technologii informatycznych pojawiła się możliwość zautomatyzowania procesu analizy partii szachowych, czemu w XX wieku zaczęli przyglądać się matematycy i informatycy.

Za pionierów w tej dziedzinie uważany jest amerykański matematyk Claude Shannon, który w roku 1950 opublikował pracę o teoretycznych aspektach programowania silników szachowych, opartych o ocenę heurystyczną oraz algorytm min-max.
Istotny wkład w rozwój szachowej sztucznej inteligencji miał także ojciec informatyki, Alan Turing, który zaprojektówał pierwszy program komputerowy w pełni zdolny do gry w szachy.
Ograniczenia techniczne tamtych czasów nie pozwoliły jednak na przetestowanie programu w praktyce.
Końcówka historii o silnikach

Silniki szachowe współcześnie

Rozwój AI i silniki szachowe


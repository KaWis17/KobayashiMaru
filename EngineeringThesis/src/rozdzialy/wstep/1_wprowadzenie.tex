\section{Wprowadzenie \colorbox{green}{Written}}
\label{sec:wprowadzenie}

Szachy, powszechnie nazywane grą królewską, to jedna z najstarszych, a zarazem najpopularniejszych form intelektualnej rozrywki w dziejach ludzkości.
Niekwestionowaną popularność tak wśród profesjonalistów, jak i amatorów zawdzięczają połączeniu prostoty zasad ze złożoności strategicznych wyzwań.
Ich historia, sięgająca VI wieku p.n.e., obejmuje stale ponawiane próby udoskonalania reguł i odkrywania nowych, coraz bardziej zaawansowanych, taktyk mających zagwarantować zwycięstwo nad oponentem.
Wprowadzenie roszady, ruchu en-passant to najbardziej spektakularne przykłady zmian, świadczących o nieograniczonej kreatywności kolejnych pokoleń graczy.


W ponad tysiąc pięćsetletniej historii szachów, szczególne znaczenie, miało upowszechnienie w połowie XX wieku zaawansowanych maszyn liczących.
Otworzyło ono możliwość zautomatyzowania procesu analizy partii szachowych.


Za pioniera w tej dziedzinie uważa się amerykańskiego matematyka Claude Shannon, który w roku 1950 opublikował pracę o teoretycznych aspektach programowania silników szachowych, opartych o ocenę heurystyczną oraz algorytm min-max.
Istotny wkład w rozwój szachowej sztucznej inteligencji miał także ojciec informatyki — Alan Turing, który rok po publikacji pracy Shannona zaprojektował pierwszy program komputerowy, w pełni zdolny do gry w szachy.
Ograniczenia techniczne tamtych czasów nie pozwoliły jednak na przetestowanie programu na maszynie.
Rozegrano niewielką liczbę partii szachowych, w których każdy ruch był obliczany analogowo.


Najstarszy program uruchomiony na komputerze, który pozwalał na przeprowadzenie pełnej rozgrywki, powstał w 1958 roku.
Od tamtego momentu wiele silników szachowych biło rekordy swoich poprzedników.
Do przełomu doszło zimą 1997 roku, kiedy to silnik szachowy DeepBlue wygrał pojedynek $3\frac{1}{2} - 2\frac{1}{2}$ z ówczesnym mistrzem świata, Garrym Kasparovem.


Po tym wydarzeniu świat wkroczył w erę super silników.
Szachy stały się nie tylko areną dla ludzkiego intelektu, ale także polem testowym zaawansowanych technologii.
Współcześnie, wykorzystanie komputerów stanowi nieodłączny element analizy partii szachowych.
Zastosowanie najnowocześniejszych rozwiązań, takich jak uczenie maszynowe i sieci neuronowe, zrewolucjonizowało sposób, w jaki rozumiemy tą gre, oraz pokazało, jak wiele jeszcze można w tej dziedzinie osiągnąć.


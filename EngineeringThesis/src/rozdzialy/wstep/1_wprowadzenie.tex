\section{Wprowadzenie}
\label{sec:wprowadzenie}

Szachy, powszechnie nazywane grą królewską, są jedną z najstarszych, a zarazem najpopularniejszych form intelektualnej rozrywki w dziejach ludzkości.
Swoją niekwestionowaną reputację, tak wśród profesjonalistów, jak i amatorów, zawdzięczają połączeniu prostoty zasad ze złożonością strategicznych wyzwań.
Ich historia, sięgająca VI~wieku p.n.e., obejmuje stale ponawiane próby udoskonalania reguł i odkrywania nowych, coraz bardziej zaawansowanych, taktyk mających zagwarantować zwycięstwo nad oponentem.
Wprowadzenie roszady, czy ruchu en-passant to najbardziej spektakularne przykłady zmian, świadczących o nieograniczonej kreatywności kolejnych pokoleń graczy.


Jednak największą zmianę w swojej ponad dwu i pół tysiącletniej historii, szachy zawdzięczają postępowi technologicznemu połowy XX wieku.
Rozwój zaawansowanych maszyn liczących otworzył możliwość zautomatyzowania procesu analizy partii szachowych na~niespotykaną dotąd skalę.


Za pioniera w tej dziedzinie uważa się amerykańskiego matematyka Claude Shannona, który w~roku 1950 opublikował pracę o teoretycznych aspektach programowania silników szachowych, opartych o ocenę heurystyczną oraz algorytm min-max.
Istotny wkład w~rozwój szachowej sztucznej inteligencji miał także ojciec informatyki – Alan Turing, który zaprojektował pierwszy program komputerowy, w pełni zdolny do rozegrania partii szachowej.
Ograniczenia techniczne tamtych czasów nie pozwoliły jednak na przetestowanie programu na maszynie.
Rozegrano niewielką liczbę partii szachowych, w~których każdy następny ruch był~obliczany przez człowieka.


Najstarszy program uruchomiony na komputerze, który pozwalał na przeprowadzenie pełnej rozgrywki, powstał w 1957 roku.
Od tamtego momentu wiele silników szachowych biło rekordy swoich poprzedników.
Do kluczowego przełomu doszło zimą 1997 roku, kiedy to~silnik szachowy DeepBlue wygrał pojedynek $3\frac{1}{2} - 2\frac{1}{2}$ z ówczesnym mistrzem świata, Garrim Kasparovem.


Po tym wydarzeniu świat wkroczył w erę super silników.
Szachy stały się nie tylko areną dla ludzkiego intelektu, ale także polem testowym zaawansowanych technologii.
Współcześnie, wykorzystanie komputerów stanowi nieodłączny element analizy partii szachowych.
Zastosowanie najnowocześniejszych rozwiązań, takich jak uczenie maszynowe i~sieci neuronowe, zrewolucjonizowało sposób, w jaki rozumiemy tą gre, oraz pokazało, jak~wiele jeszcze można w tej dziedzinie osiągnąć.


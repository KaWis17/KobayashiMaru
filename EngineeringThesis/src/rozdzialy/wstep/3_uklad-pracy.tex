\section{Układ pracy}
\label{sec:uklad-pracy}

Niniejsza praca inżynierska składa się z trzech głównych części.

W pierwszej z nich opisano elementy oprogramowania konieczne do stworzenia podstawowej wersji silnika szachowego.
Przedstawiono metody komunikacji z interfejsem, techniki reprezentacji pozycji i generowania dostępnych ruchów, algorytm minimalizowania start i~maksymalizowania zysków wraz z oceną heurystyczną oraz metodologię zarządzania czasem gry.
Przetestowano aplikację w celu sprawdzenia poprawności działania.

Następny rozdział poświęcono pracy nad ulepszeniem systemu.
Przedstawiono zaimplementowane rozwiązania mające na celu poprawę poziomu gry systemu.
Skupiono się na dwóch kierunkach: poprawie prędkości przeszukiwania stanów oraz na udoskonaleniu precyzji heurystycznej oceny pozycji.
Jako że dla słabych silników o wiele większe znaczenie w~poprawie poziomu jego gry ma pierwszy z tych aspektów, omówiono go najpierw~\cite*{Vrzina2023}.
Z uwagi na mnogość opisanych w literaturze możliwych rozwiązań danego problemu algorytmicznego, w niektórych miejscach przedstawiono także techniki, które ostatecznie nie zostały zaimplementowane w silniku, a także wytłumaczono, czym kierował się autor przy wyborze danego rozwiązania.

Ostatnią część pracy poświęcono testom wydajnościowym oraz jakościowym.
Przetestowano program pomiędzy różnymi jego wersjami.
Przedstawiono wyniki gry systemu przeciwko innym, publicznie dostępnym silnikom.
Na końcu podsumowano rozgrywki przeprowadzone z graczami.
W dodatku do pracy zamieszczono instrukcję wdrożenia aplikacji w celu odpalenia silnika we własnym środowisku.

Choć autor zakłada, że czytelnik zna zasady gry w szachy, w pracy zawarto także omówienie niektórych, bardziej skomplikowanych bądź mniej znanych, jej aspektów.
Wiele z opisanych rozwiązań algorytmicznych można znaleźć jedynie w~literaturze anglojęzycznej.
W miejscach, gdzie tłumaczenie uznano za niewystarczające, podano także oryginalne nazewnictwo.
\usepackage{biblatex}\section{Układ pracy}
\label{sec:uklad-pracy}

Niniejsza praca inżynierska składa się z trzech głównych części.

W pierwszej z nich opisano elementy oprogramowania konieczne do stworzenia podstawowej wersji silnika szachowego.
Przedstawiono metody komunikacji z interfejsem, techniki reprezentacji pozycji i generowania dostępnych ruchów, algorytm minimalizowania start i~maksymalizowania zysków wraz z oceną heurystyczną oraz metodologią zarządzania czasem gry.
Przetestowano aplikację w celu sprawdzenia poprawności działania.

Następny rozdział poświęcono pracy nad ulepszeniem systemu.
Przedstawiono zaimplementowane rozwiązania mające na celu poprawę poziomu gry systemu.
Skupiono się na dwóch kierunkach: poprawy prędkości przeszukiwania stanów oraz na poprawie precyzji heurystycznej oceny pozycji.
Jako że dla słabych silników o wiele większe znaczenie w poprawie jego gry ma pierwszy z tych aspektów, omówiono go w pierwszej kolejności. \cite*{Vrzina2023}

Ostatnią część pracy poświęcono testom wydajnościowym oraz jakościowym.
Przetestowano program pomiędzy różnymi jego wersjami.
W dalszej części przedstawiono wyniki gry systemu przeciwko innym, publicznie dostępnym silnikom.
Na końcu podsumowano rozgrywki przeprowadzone z graczami.

W dodatku do pracy zamieszczono instrukcję wdrożenia aplikacji w celu odpalenia silnika we własnym środowisku.

Choć autor zakłada, że czytelnik zna zasady gry w szachy, w pracy zawarto także omówienie niektórych, bardziej skomplikowanych bądź mniej znanych, aspektów tej gry.
Wiele z opisanych w niniejszej pracy rozwiązań algorytmicznych można znaleźć jedynie w literaturze anglojęzycznej, a ich przetłumaczenie nie jest dokładne.
\section{Układ pracy \colorbox{green}{Written}}
\label{sec:uklad-pracy}

Niniejsza praca inżynierska składa się z trzech głównych części.

W pierwszej z nich opisano elementy oprogramowania konieczne do stworzenia podstawowej wersji silnika szachowego.
Przedstawiono metody komunikacji z interfejsem, techniki reprezentacji pozycji, algorytm min-max wraz z metodologią zarządzania czasem gry.
Przetestowano aplikację w celu sprawdzenia poprawności działania.

Następny rozdział poświęcono pracy nad ulepszeniem systemu.
Przedstawiono zaimplementowane rozwiązania, mające na celu poprawę poziomu gry systemu.

Ostatnią część pracy poświęcono testom wydajnościowym oraz jakościowym.
Przetestowano w pierwszej kolejności program pomiędzy różnymi jego wersjami.
W dalszej kolejności przetestowano system przeciwko innym, publicznie dostępnym silnikom.
Na końcu przedstawiono podsumowanie rozgrywek przeprowadzonych z graczami.

W dodatku do pracy zamieszczono instrukcję wdrożenia aplikacji w celu odpalenia silnika we własnym środowisku.
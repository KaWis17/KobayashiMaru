\section{Układ pracy}
\label{sec:uklad-pracy}

Niniejsza praca inżynierska składa się z trzech głównych części.

W pierwszej z nich opisano elementy oprogramowania konieczne do stworzenia podstawowej wersji silnika szachowego.
Przedstawiono metody komunikacji z interfejsem, techniki reprezentacji pozycji, algorytm min-max wraz z metodologią zarządzania czasem gry.
Przetestowano aplikację w celu sprawdzenia poprawności działania.
Choć do zrozumienia pracy konieczna jest znajomość zasad gry w szachy, to w tym rozdziale mówiono także niektóre, bardziej zawiłe bądź mniej znane, jej aspekty.

Następny rozdział poświęcono pracy nad ulepszeniem systemu.
Przedstawiono zaimplementowane rozwiązania, mające na celu poprawę poziomu gry systemu.
Skupiono się na dwóch kierunkach: poprawy prędkości przeszukiwania oraz na poprawie precyzji oceny heurystycznej.
Jako że dla słabych silników o wiele większe znaczenie w poprawie jego gry ma pierwszy z tych aspektów, omówiono go na początku.


Ostatnią część pracy poświęcono testom wydajnościowym oraz jakościowym.
Przetestowano w pierwszej kolejności program pomiędzy różnymi jego wersjami.
W dalszej kolejności przetestowano system przeciwko innym, publicznie dostępnym silnikom.
Na końcu przedstawiono podsumowanie rozgrywek przeprowadzonych z graczami.

W dodatku do pracy zamieszczono instrukcję wdrożenia aplikacji w celu odpalenia silnika we własnym środowisku.
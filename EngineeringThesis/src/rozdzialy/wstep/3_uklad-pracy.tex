\section{Układ pracy}
\label{sec:uklad-pracy}

Niniejsza praca inżynierska składa się z trzech głównych części.

W pierwszej z nich opisano elementy oprogramowania konieczne do stworzenia podstawowej wersji silnika szachowego.
Przedstawiono metody komunikacji z interfejsem, techniki reprezentacji pozycji i generowania dostępnych ruchów.
Zaprezentowano algorytm minimalizowania start i~maksymalizowania zysków wraz z oceną heurystyczną oraz metodologię zarządzania czasem gry.
Omówiono niektóre z testów jednostkowych stworzonych w celu sprawdzenia poprawności działania aplikacji.

Następny rozdział poświęcono pracom nad ulepszeniem systemu.
Przedstawiono zaimplementowane rozwiązania mające na celu poprawę poziomu gry.
Skupiono się~na~dwóch kierunkach: poprawie prędkości przeszukiwania stanów oraz na udoskonaleniu precyzji heurystycznej oceny pozycji.
Z uwagi na to, że dla słabych silników o wiele większe znaczenie w~poprawie poziomu gry ma pierwszy z tych aspektów, omówiono go w pierwszej kolejności\cite*{Vrzina2023}.
W~literaturze można znaleźć więcej niż jedno możliwe podejście do niektórych z opisanych problemów algorytmicznych.
Ze względu na ich mnogość, w pracy przedstawiono parę możliwych rozwiązań, a następnie wskazano, czym kierował się autor przy wyborze jednego z~nich.

Ostatnią część pracy poświęcono testom wydajnościowym oraz jakościowym.
Przetestowano program pomiędzy różnymi jego wersjami.
Przedstawiono wyniki gry systemu przeciwko innym, publicznie dostępnym silnikom.
W dodatkach do pracy zamieszczono instrukcję wdrożenia aplikacji w celu odpalenia silnika we własnym środowisku.

Choć autor zakłada, że czytelnik zna zasady gry w szachy, w pracy zawarto także omówienie niektórych, bardziej skomplikowanych bądź mniej znanych, jej aspektów.

Wiele z opisanych rozwiązań algorytmicznych można znaleźć jedynie w~literaturze anglojęzycznej.
W miejscach, gdzie tłumaczenie uznano za niewystarczające, podano także oryginalne nazewnictwo.
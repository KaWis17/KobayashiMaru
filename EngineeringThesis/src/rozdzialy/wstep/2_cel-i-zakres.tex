\section{Cel i zakres}
\label{sec:cel-i-zakres}

Zasadniczym celem pracy było stworzenie silnika szachowego, zdolnego do oceny heurystycznej pozycji, oraz proponowania ruchów z uwzględnieniem ich strategicznych aspektów.
Zakres pracy objął następujące zagadnienia:
\begin{itemize}
    \item Przegląd literatury na temat technik oraz algorytmów wykorzystywanych przy tworzeniu nowoczesnych silników szachowych.
    \item Zapoznanie się z powszechnie obowiązującymi zasadami turniejowej gry w szachy, opublikowanymi przez Międzynarodową Federację Szachową.
    \item Stworzenie silnika szachowego w języku programowania Java 22, z celowym pominięciem dodatkowych rozwiązań open-source.
    \item Wykorzystanie Uniwersalnego Interfejsu Szachowego do komunikacji z systemem.
    \item Zintegrowanie silnika z wybranym interfejsem graficznym.
    \item Testowanie poprawności stworzonego oprogramowania.
    \item Implementacja rozwiązań programistycznych przyspieszających przeszukiwanie drzewa decyzyjnego oraz ulepszających dokładność oceny heurystycznej.
    \item Przeprowadzenie analizy porównawczej pomiędzy wersjami systemu w celu oceny efektywności zastosowanych rozwiązań.
    \item Porównanie najlepszej wersji silnika z już istniejącymi rozwiązaniami w celu określenia poziomu gry.
\end{itemize}


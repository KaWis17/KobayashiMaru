\subsection{Uniwersalny Interfejs Szachowy \colorbox{cyan}{In progress}}
\label{subsec:interfejs-szachowy}

Uniwersalny Interfejs Szachowy (ang.~\emph{Universal Chess Interface}, w~skrócie UCI) jest ustandaryzowanym protokołem tekstowym, służącym do wymiany informacji pomiędzy różnymi programami szachowymi.
Jego implementacja pozwoliła na komunikację silnika szachowego z wybranymi interfejsami graficznymi oraz środowiskami testowymi.

UCI jest protokołem rozbudowanym, pozwalającym między innymi na rozgrywki innych wersji szachów niż europejskie, dla przykładu Chess960.
W silniku zaimplementowano jedynie te z komend, które wystarczyły do rozegrania podstawowej partii mierzonej czasowo.


\begin{table}[htb] \small
\centering
\caption{UCI - komunikacja GUI do silnika}
\label{tab:UCI_GUI_silnik}
\begin{tabularx}{\linewidth}{|p{.25\linewidth}|X|}\hline
    Komenda & Opis działania \\ \hline\hline

    uci & Wytłumaczenie działania komendy \\ \hline
    debug & Wytłumaczenie działania komendy \\ \hline
    isready & Wytłumaczenie działania komendy \\ \hline
\end{tabularx}
\end{table}

\begin{table}[htb] \small
\centering
\caption{UCI - komunikacja silnika do GUI}
\label{tab:UCI_silnik_GUI}
\begin{tabularx}{\linewidth}{|p{.25\linewidth}|X|}\hline
Komenda & Opis działania \\ \hline\hline

id & Wytłumaczenie działania komendy \\ \hline
uciok & Wytłumaczenie działania komendy \\ \hline
readyok & Wytłumaczenie działania komendy \\ \hline
\end{tabularx}
\end{table}

Metodę połączenia z dowolnym programem obsługującym UCI przedstawiono w dodatku \ref{ch:instrukcja-wdrozenia}.

Przykład wymiany informacji pomiędzy aplikacją a GUI zaprezentowano w dodatku \ref{ch:przyklad-uci}.
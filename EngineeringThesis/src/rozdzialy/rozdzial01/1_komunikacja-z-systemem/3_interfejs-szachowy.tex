\subsection{Uniwersalny Interfejs Szachowy}
\label{subsec:interfejs-szachowy}

Uniwersalny Interfejs Szachowy (ang.~\emph{Universal Chess Interface}, w~skrócie UCI) \cite*{UCIdoc} jest~ustandaryzowanym protokołem tekstowym, służącym do wymiany informacji pomiędzy różnymi programami szachowymi.
Jego implementacja pozwoliła na komunikację silnika szachowego z wybranymi interfejsami graficznymi oraz środowiskami testowymi.

UCI jest protokołem rozbudowanym, pozwalającym między innymi na rozgrywki innych wersji szachów niż europejskie, dla przykładu Chess960.
W silniku zaimplementowano jedynie te z komend, które konieczne były do rozegrania podstawowej partii mierzonej czasowo.


Metodę połączenia z dowolnym programem obsługującym UCI przedstawiono w dodatku \ref{ch:instrukcja-wdrozenia}.
Opisy komend oraz przykład wymiany informacji pomiędzy aplikacją a GUI zaprezentowano w~dodatku \ref{ch:protokol-uci}.
\subsection{Szachowa Notacja Algebraiczna}
\label{subsec:notacja-algebraiczna}

Kluczowym aspektem, z perspektywy komunikacji z systemem, jest także określenie formatu zapisu ruchów.
W aplikacji wykorzystano Szachową Notację Algebraiczną.

Notacja ta, w swojej krótkiej formie, jest powszechnie stosowana w literaturze oraz podczas oficjalnych zawodów.
Zawiera informacje o rodzaju ruszanej bierki oraz o jej polu docelowym.
Zapis ten z punktu widzenia komputerów zawiera jednak wadę.
W sytuacji, w której dwie bierki tego samego rodzaju mogą poruszyć się na jedno pole, występuje dwuznaczność zapisu~\ref{fig: basic_chess_positions}.
Choć w takiej sytuacji dodaje się do ruchu kolumnę bądź wiersz startowy różniący obie bierki, jest to rozwiązanie wymagające implementacji dodatkowej logiki, oraz wiedzy o stanie całej planszy.

Znacznie bardziej intuicyjne dla komputerów jest zastosowanie długiej wersji szachowej notacji algebraicznej.
Zawarte są w niej informacje o polu startowym oraz polu docelowym ruchu, usuwając tym samym ryzyko dwuznaczności.
Roszady oznaczano przez pola ruchu króla, natomiast do ruchów z promocją dopisano literę określającą rodzaj podmienionej figury.


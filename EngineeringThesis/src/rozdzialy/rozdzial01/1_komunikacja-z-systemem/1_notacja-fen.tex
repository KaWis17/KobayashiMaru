\subsection{Notacja Forsytha-Edwardsa \colorbox{green}{Written}}
\label{subsec:notacja-fen}


W 1950 roku amerykański matematyk Claude Shanon na łamach \("\)Philosophical Magazine\("\) opublikował pracę naukową zatytułowaną \("\)Programowanie komputera do gry w szachy\("\). \cite*{Shannon1950XXIIPA}
Stała się ona teoretyczną podstawą dla dalszego rozwoju silników szachowych.
Zawarte w niej zostało między innymi oszacowanie, co do ilości możliwych pozycji szachowych, wynoszące~$10^{43}$.
Oznacza to, że liczba legalnych ułożeń planszy przewyższa o rzędy wielkości liczbę gwiazd w widzialnym wszechświecie.


Aby umożliwić użytkownikowi wprowadzenie danych oraz komunikację z programem, należało w pierwszej kolejności sprecyzować format, w jakim zostaną dostarczone informacje dotyczące aktualnej pozycji.
Standardem, wykorzystywanym nie tylko w większości silników, ale także w pojedynkach rozgrywanych online, jest Notacja Forsytha-Edwardsa (ang.~\emph{Forsyth–Edwards~Notation}, w~skrócie FEN).
Stworzona pierwotnie przez dziennikarza Davida Forsytha, a następnie dostosowana do potrzeb komputerów przez Stevena Edwardsa.

Notacja FEN jest linią znaków ASCII, która pozwala na jednoznaczne określenie aktualnego stanu gry.
Wielkimi literami kodowane są bierki białe, małymi natomiast bierki czarne.
Każda z nich określona jest skrótem pochodzącym od ich angielskich nazw:
\begin{multicols}{3}
    \begin{itemize}
        \item P/p — pion
        \item N/n — skoczek
        \item B/b — goniec
        \item R/r — wieża
        \item Q/q — hetman
        \item K/k — król
    \end{itemize}
\end{multicols}

Sześć następujących po sobie pól, oddzielonych spacjami, określa następujące aspekty gry:
\begin{enumerate}
    \item Reprezentacja 64 pól szachownicy z perspektywy białego gracza.
    Każdy z wierszy planszy oddzielony jest \("\)/\("\), a jego zawartość opisywana zostaje od kolumny a, do kolumny h.
    Liczbę nieprzerwanie pustych pól w danym wierszu określa cyfra z zakresu od 1 do 8.
    \item Sprecyzowanie, do którego gracza należy następny ruch (w — biały, b — czarny).
    \item Przedstawienie możliwości roszady obu stron (K/k — krótka roszada, Q/q — długa roszada).
    \item Sprecyzowanie pola będącego celem bicia w przelocie, szerzej znanego jako ruch en~passant.
    Brak możliwości bicia określany jest jako\("\)-\("\)
    \item Liczba posunięć od ostatniego bicia bądź ruchu pionem.
    Wartość ta jest istotna z punktu widzenia reguły 50 posunięć.
    \item Liczba pełnych ruchów, która zostaje każdorazowo zwiększana po ruchu czarnych bierek.
\end{enumerate}

Pełna specyfikacja FEN dostępna jest w dokumentacji \("\)Portable Game Notation\("\). \cite*{PGNdoc}

Poniżej przedstawiono przykładowe FEN, wraz z korespondującymi im ułożeniami szachownicy:

\vspace{5mm}
\centerline{
    \ref{fig: figure} a) \lstset{basicstyle=\ttfamily}\lstinline{rnbqkbnr/pppppppp/8/8/8/8/PPPPPPPP/RNBQKBNR w KQkq - 0 1}
}
\centerline{
    \ref{fig: figure} b) \lstset{basicstyle=\ttfamily}\lstinline{rnbqkbnr/pppppppp/8/8/4P3/8/PPPP1PPP/RNBQKBNR b KQkq e3 0 1}
}
\vspace{5mm}

\begin{figure}[htb]
    \centering
    \begin{tabular}{@{}ll@{}}
        a) & b) \\
        \includegraphics[width=0.475\textwidth]{rozdzialy/rozdzial01/1_komunikacja-z-systemem/rysunki/pozycja_startowa}
        &
        \includegraphics[width=0.475\textwidth]{rozdzialy/rozdzial01/1_komunikacja-z-systemem/rysunki/pozycja_startowa_e2e4}
    \end{tabular}
    \caption{Przykładowe pozycje szachowe: a) startowa, b) po ruchu e2e4}
    \label{fig: figure}
\end{figure}


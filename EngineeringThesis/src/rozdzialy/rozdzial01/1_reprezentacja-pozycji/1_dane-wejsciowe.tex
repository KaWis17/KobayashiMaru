\subsection{Dane wejściowe}
\label{subsec:dane-wejsciowe}

\subsubsection{Notacja Forsytha-Edwardsa}

Aby móc reprezentować dowolną pozycję szachową, należy w pierwszej kolejności określić, w jakim formacie zostanie ona do programu.
Standardem wykorzystywanym w większości silników szachowych jest Notacja Forsytha-Edwardsa (ang.~\emph{Forsyth–Edwards~Notation}, w skrócie FEN), stworzona specjalnie na potrzeby komputerów.
Pozwala ona na jednoznaczne określenie pozycji na szachownicy.
Przykład FEN dla pozycji startowej:

\vspace{5mm}
\centerline{
    \lstset{basicstyle=\ttfamily}\lstinline{rnbqkbnr/pppppppp/8/8/8/8/PPPPPPPP/RNBQKBNR w KQkq - 0 1}
}
\vspace{5mm}

Całość informacji, zawartych w jednej linii znaków kodowanych ASCII, ma 6 pól oddzielonych spacjami.
Pola te informują, o różnych aspektach danej pozycji:

\begin{itemize}
    \item bia�y,
    \item niebieski,
    \item czerwony.
\end{itemize}

\subsubsection{Szachowa notacja algebraiczna}
asd
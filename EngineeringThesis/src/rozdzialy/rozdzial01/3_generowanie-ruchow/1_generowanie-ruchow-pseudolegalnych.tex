\subsection{Generowanie ruchów pseudolegalnych \colorbox{cyan}{In progress}}
\label{subsec:generowanie-ruchow-pseudolegalnych}

\subsubsection{Generowanie ruchów piona}

Pionek na szachownicy ma do dyspozycji parę możliwych ruchów, które zależą od aktualnej pozycji na planszy:
\begin{itemize}
    \item ruch o jedno pole do przodu, jeśli to pole to jest wolne,
    \item ruch o dwa pola do przodu, jeśli pionek znajduje się na swoim polu startowym oraz zarówno pole docelowe, jak i pole, przez które przechodzi, są wolne,
    \item bicie w przód na skos w lewo lub prawo, jeśli na polu docelowym znajduje się bierka przeciwnika,
    \item bicie en-passant.
\end{itemize}
Powyższe ruchy zostały zaimplementowane przez operacje bitowe i bitowe przesunięcia na masce bitowej reprezentującej pionki.
Dla przykładu technika generowania ruchów piona o jedno pole do przodu:



\subsubsection{Generowanie ruchów hetmana, wieży i gońca}

\subsubsection{Generowanie ruchów króla i skoczka}


\section{Generowanie ruchów}
\label{sec:generowanie-ruchow}

Generowanie możliwych ruchów w danej pozycji jest jednym z podstawowych, ale~też~kluczowych elementów każdego silnika szachowego.
Jego efektywna implementacja to taka, w której silnik spędza jak najmniej czasu, pozostawiając możliwości obliczeniowe na przeszukiwanie drzewa decyzyjnego.

Podstawowym rozróżnieniem stosowanych rozwiązań jest podział na generowanie ruchów pseudolegalnych oraz ruchów legalnych.
Ruch pseudolegalny to taki, który nie narusza zasad ruchów poszczególnych bierek, natomiast istnieje ryzyko, że po jego wykonaniu własny król znajdzie się w szachu.
Takie rozwiązanie jest możliwe do zaimplementowania, z~uwagi na pozostawienie odpowiedzialności za sprawdzenie legalności ruchu funkcji, która ten ruch wykonuje.
Główną zaletą tego rozwiązania jest znacznie łatwiejsza implementacja.

\subsection{Operacje na mapach bitowych}
\label{subsec:operacje-na-mapach-bitowych}

Jak już wcześniej wspomniano, iteracyjne generowanie ruchów każdej z bierek z osobna mogłoby okazać się zbyt czasochłonne.
Z tego względu większość dostępnych posunięć tworzona jest dzięki przekształcaniu map bitowych reprezentujących konkretny typ figury.
Optymalizacja przynosi szczególny efekt przy generowaniu ruchów pionów, ze względu na~ich~znaczną liczbę przez większość czasu gry.

Pion ma do dyspozycji kilka możliwych ruchów, które należało zaimplementować: ruch o~jedno pole do przodu, ruch o dwa pola do przodu, bicie w lewo i bicie w prawo.

Maski dla każdego z tych ruchów zostały wygenerowane w oddzielnych metodach.
Przykładowo, dla ruchu o dwa pola do przodu wzór wygląda następująco:
\begin{align*}
    moves & = empty && \text{pole docelowe musi być puste} \\
    moves & = moves \wedge (pionki_w\ll16) && \text{biały pion musi być dwa wiersze niżej}\\
    moves & = moves \wedge (empty\ll8) && \text{wiersz niżej musi być pusty}\\
    moves & = moves \wedge rank4 && \text{pole docelowe musi być w czwartym wierszu}
\end{align*}

W taki sposób stworzono maskę bitową końcowych pól, na które piony mogą się przesuwać, skacząc o dwa pola.
Na otrzymanym wyniku należy przeprowadzić serializację, to jest przekształcić go na listę dostępnych ruchów.
Aby nie iterować przez wszystkie 64 bity, zastosowano technikę zwaną ang. \emph{Bit Scan}, która zwraca indeks najbardziej istotnego bitu na~masce, dodaje ruch do listy, a następnie usuwa ten bit z maski.
Operacja jest wykonywana do~momentu, aż maska pozostanie pusta.
\begin{lstlisting}[
    language=Java,
    style=JavaStyle,
    caption=Metoda dodająca ruchy z maski wraz z przykładowym wywołaniem,
    label=lst:mask]
     addMovesFromMask(movesMask, moveType, offset) {
        while(movesMask != 0L) {
            index = (64 - Long.numberOfLeadingZeros(movesMask));
            possibleMoves.add(new Move(index+offset, index, moveType));
            movesMask &= ~(1L << (index - 1));
        }
     }

    addMovesFromMask(allDoublePushMask, DOUBLE_PAWN_PUSH, -16);

\end{lstlisting}

Legalne ruchy króla i skoczka generowane są w sposób analogiczny, z tą jednak różnicą, że maski dostępnych ruchów tworzone są nie przez przesunięcia bitowe, ale przy inicjalizacji silnika generowana jest tablica dla każdego z pól startowych.

%
%\subsubsection{Generowanie ruchów hetmana, wieży i gońca}
%
%Ruchy hetmana są połączeniem ruchów wieży oraz gońca, z tego względu można je generować w ten sam sposób.
%Techniki te operują na bardzo podobnych zasadach, co generowanie ruchów piona, z tą różnicą, że figury mogą poruszać się o dowolną liczbę pól w danym kierunku, aż do momentu napotkania innej bierki na swojej drodze.
%Aby uniknąć skomplikowanych obliczeń, należało zaimplementować funkcję, która w literaturze znana jest pod nazwą (ang.~\emph{Hyperbola Quintessence}).
%
%\subsubsection{Generowanie ruchów króla i skoczka}


%\begin{figure}[ht]
%    \centering
%    \includegraphics[width=0.85\linewidth]{rozdzialy/rozdzial01/3_generowanie-ruchow/rysunki/bitboards-arithmetic}
%    \caption{Kodowanie ruchu szachowego}
%    \label{fig:bitboards-arithmetic}
%\end{figure}
\subsection{Hyperbola Quintessence}
\label{subsec:hyperbola-quintessence}

Generowanie ruchów hetmana, wieży i gońca odbywa się w sposób odmienny.
Wynika to~z~faktu, że figury te poruszają się~o~dowolną liczbę pól w danym kierunku, aż do momentu napotkania innej bierki na swojej drodze.
W przypadku bierki przeciwnika możliwe jest bicie, w przypadku bierki własnej, należy zatrzymać się o pole wcześniej.

Choć są to trzy różne figury, to mają do dyspozycji dwa możliwe ruchy, ruch w linii prostej, jak wieża, oraz ruch po przekątnej, jak goniec.
Ruchy hetmana są natomiast sumą dwóch powyższych generatorów.

Większość silników szachowych korzystających z masek bitowych implementuje funkcję, która w literaturze zwana jest jako ang. \emph{Hyperbola Quintessence}.
Pozwala ona na wygenerowanie dostępnych ruchów w jednej płaszczyźnie.
\begin{align*}
    o = \text{11010101} && o' = \text{10101011} && \text{pola zajęte przez bierki} \\
    r = \text{00010000} && r' = \text{00001000} && \text{pole figury dla której generujemy ruchy} \\
    o-r = \text{11000101} && o'-r' = \text{10100011} && \text{pola zajęte minus pole figury} \\
    \alpha = o-2r = \text{10110101} && \beta = o'-2r' = \text{10011011} && \text{pola zajęte dwukrotnie minus pole figury} \\
    (\alpha \oplus \beta') \wedge \neg \gamma && \text{01101100} && \text{maska legalnych ruchów}
\end{align*}
\begin{multicols}{2}
    \begin{itemize}[label={}]
        \item \(\oplus\) — alternatywa wykluczająca
        \item \(\wedge\) — koniunkcja
        \item \(x'\) — odwrócenie bitów
        \item \(\gamma\) — maska pól zajętych przez własne bierki
    \end{itemize}
\end{multicols}




\subsection{Ruchy specjalne}
\label{subsec:ruchy-specjalne}

Pozostałymi ruchami do zaimplementowania były bicie w przelocie oraz roszady.
Oba z nich wymagały dodatkowej weryfikacji z danymi dostępnymi w reprezentacji stanu gry.

Pierwszy z nich - en-passant - został wygenerowany przez nałożenie na siebie maski pola bicia w przelocie oraz maski pinów, odpowiednio przesuniętych o $\pm 7$ oraz $\pm 9$ kratek.

Aby uzyskać prawo roszady, należy spełnić następujące warunki:
\begin{enumerate}
    \item Ani król, ani wieża biorąca udział w roszadzie nie mogły wykonać wcześniej ruchu.
    \item Pola między królem a wieżą muszą być puste.
    \item Król nie może znajdować się w szachu.
    \item Król nie może przechodzić bądź kończyć ruch na polach, atakowanych przez bierki przeciwnika.
\end{enumerate}




\subsection{Generowanie ruchów legalnych}
\label{subsec:generowanie-ruchow-legalnych}

\subsubsection{Technika usuwania ruchów pseudo-legalnych}

Po zaimplementowaniu logiki opisanej powyżej, silnik był zdolny do generowania posunięć pseudolegalnych.
Natomiast ruch, po którym własny król znajduje się w szachu, nie tylko jest ruchem nieoptymalnym, ale również z punktu widzenia reguł FIDE nielegalnym.

W literaturze można znaleźć kilka podejść do problemu odfiltrowywania ruchów pseudolegalnych.
Niektóre z nich korzystają z dodatkowych masek bitowych, reprezentujących pola atakowane przez bierki danej strony.
W niniejszej implementacji zastosowano jednak rozwiązanie, w subiektywnym odczuciu autora, łatwiejsze.

Po wykonaniu konkretnego ruchu, w miejscu, gdzie znajduje się król, stawiane są kolejne figury oraz generowane są dostępne bicia.
Jeśli wśród bić znajduje się bierka przeciwnika, tego samego typu, co aktualnie podstawiona, oznacza to, że król znajduje się w szachu, a więc posunięcie nie należy do kategorii ruchów legalnych.

\subsubsection{Test wydajności}

Generatory ruchów szachowych posiadają skomplikowaną logikę.
Z tego względu bardzo łatwo o popełnienie błędu w implementacji.
Dopuszczenie choćby jednego błędu, skutkować będzie jego propagacją na większych głębokościach, a w skrajnych przypadkach doprowadzi do~zakończenia programu.

Ocena poprawności metodą przeprowadzania rozgrywek z silnikiem szachowym jest~rozwiązaniem czasochłonnym.
Odwiedzenie dużej ilości węzłów drzewa gry, w~celu~sprawdzenia poprawności wykonywania ruchów, jest praktycznie niemożliwe.
Błędy w~ten~sposób powstałe są trudne do zlokalizowania.

Z tego względu zastosowano Test Wydajności (ang.~\emph{Performance Testing}, w~skrócie Perft~Test).
Choć nazwa mogłaby wskazywać na testowanie prędkości generowanych ruchów, test ten można przeprowadzić także w celu kontroli poprawności.
Metoda ta~opiera się~na~wykorzystaniu algorytmu DFS dla ograniczonej głębokości na drzewie gry, przy jednoczesnym zliczaniu odwiedzonych węzłów.
Tak otrzymane wyniki można porównać z konsensusem osiągniętym przez twórców silników szachowych.
Jako punkt odniesienia autor przyjął wyniki generowane przez silnik Stockfish.
Przeprowadzenie testów z różnych pozycji startowych oraz dla~różnych głębokości pozwoliło na potwierdzenie poprawności implementowanego generatora, z~prawdopodobieństwem graniczącym z pewnością.
Przykładowe wyniki zaprezentowano w dodatku \ref{ch:wyniki-perft}.


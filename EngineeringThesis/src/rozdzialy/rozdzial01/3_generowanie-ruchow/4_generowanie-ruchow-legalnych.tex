\subsection{Generowanie ruchów legalnych}}
\label{subsec:generowanie-ruchow-legalnych}

\subsubsection{Technika usuwania ruchów pseudo-legalnych}

Program umożliwia ruchy pseudo-legalne.
Ruch po którym własny król znajduje się w szachu nie tylko jest nieoptymalny, ale również z punktu widzenia reguł FIDE jest nielegalny.
W celu zagwarantowania poprawności ruchów, konieczne było upewnienie się, że ruch jest legalny.

Osiągnięto to w ten sposób, że po wykonaniu ruchu, generuje się ruchy wszystkich możliwych figur przeciwnika od pola, na którym znajduje się król.
Jeśli którykolwiek z tych posunięć zakończyłby się biciem figury tego samego typu, ruch jest nielegalny.

TODO...

\subsubsection{Perft test}

Z uwagi na złożoność powyższego problemu, konieczne było wykonanie testów jednostkowych, które pozwolą sprawdzić poprawność otrzymywanych tablic ruchów.

TODO...

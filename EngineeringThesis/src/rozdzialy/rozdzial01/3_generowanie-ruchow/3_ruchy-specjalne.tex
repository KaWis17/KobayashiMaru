\subsection{Ruchy specjalne}
\label{subsec:ruchy-specjalne}

Pozostałymi ruchami do zaimplementowania było bicie w przelocie oraz cztery roszady.
Z uwagi na znikomą liczbę takich posunięć w danej pozycji oraz na złożoną logikę tych ruchów, zostały one wygenerowane explicite z zasad gry.

Oba typy ruchów wymagały dodatkowej weryfikacji z danymi dostępnymi w reprezentacji stanu gry.
Pierwszy z nich – en-passant – został wygenerowany przez nałożenie na siebie maski pola bicia w przelocie oraz maski pionów, odpowiednio przesuniętych o $\pm 7$ oraz $\pm 9$ kratek.

Aby uzyskać prawo roszady, należy spełnić następujące warunki:
\begin{enumerate}
    \item Ani król, ani wieża biorąca udział w roszadzie nie mogły wykonać wcześniej ruchu.
    \item Pola między królem a wieżą muszą być puste.
    \item Król nie może znajdować się w szachu.
    \item Król nie może przechodzić bądź kończyć ruch na polach atakowanych przez bierki przeciwnika.
\end{enumerate}




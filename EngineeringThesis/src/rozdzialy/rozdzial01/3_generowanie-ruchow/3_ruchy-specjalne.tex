\subsection{Ruchy specjalne}
\label{subsec:ruchy-specjalne}

Pozostałymi ruchami do zaimplementowania były bicie w przelocie oraz roszady.
Oba z nich wymagały dodatkowej weryfikacji z danymi dostępnymi w reprezentacji stanu gry.

Pierwszy z nich - en-passant - został wygenerowany przez nałożenie na siebie maski pola bicia w przelocie oraz maski pinów, odpowiednio przesuniętych o $\pm 7$ oraz $\pm 9$ kratek.

Aby uzyskać prawo roszady, należy spełnić następujące warunki:
\begin{enumerate}
    \item Ani król, ani wieża biorąca udział w roszadzie nie mogły wykonać wcześniej ruchu.
    \item Pola między królem a wieżą muszą być puste.
    \item Król nie może znajdować się w szachu.
    \item Król nie może przechodzić bądź kończyć ruch na polach, atakowanych przez bierki przeciwnika.
\end{enumerate}




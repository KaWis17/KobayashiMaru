\subsection{Hyperbola Quintessence}
\label{subsec:hyperbola-quintessence}

Generowanie ruchów hetmana, wieży i gońca odbywa się w sposób odmienny.
Wynika to~z~faktu, że figury te poruszają się~o~dowolną liczbę pól w danym kierunku, aż do momentu napotkania innej bierki na swojej drodze.
W przypadku bierki przeciwnika możliwe jest bicie, w przypadku bierki własnej, należy zatrzymać się pole wcześniej.
Choć są to trzy różne figury, to mają do dyspozycji dwa możliwe ruchy, ruch w linii prostej, jak wieża, oraz ruch po przekątnej, jak~goniec.
Ruchy hetmana są natomiast sumą dwóch powyższych generatorów.

Większość silników szachowych korzystających z masek bitowych implementuje funkcję, która w literaturze zwana jest jako ang. \emph{Hyperbola Quintessence}.
Pozwala ona na wygenerowanie dostępnych ruchów w jednej prostej oraz na uniknięcie czasochłonnej i nieczytelnej logiki iteracyjnej.
Poniższy przykład został zaczerpnięty ze źródła \cite{hyperbola-quintessence}.
\begin{align*}
    o = \text{11010101} && o' = \text{10101011} && \text{pola zajęte przez bierki} \\
    r = \text{00010000} && r' = \text{00001000} && \text{pole figury dla której generujemy ruchy} \\
    o-r = \text{11000101} && o'-r' = \text{10100011} && \text{pola zajęte minus pole figury} \\
    \alpha = o-2r = \text{10110101} && \beta = o'-2r' = \text{10011011} && \text{pola zajęte dwukrotnie minus pole figury} \\
    (\alpha \oplus \beta') \wedge \neg \gamma && \text{01101100} && \text{maska legalnych ruchów}
\end{align*}
\begin{multicols}{2}
    \begin{itemize}[label={}]
        \item \(\oplus\) — alternatywa wykluczająca
        \item \(\wedge\) — koniunkcja
        \item \(x'\) — odwrócenie bitów
        \item \(\gamma\) — maska pól zajętych przez własne bierki
    \end{itemize}
\end{multicols}





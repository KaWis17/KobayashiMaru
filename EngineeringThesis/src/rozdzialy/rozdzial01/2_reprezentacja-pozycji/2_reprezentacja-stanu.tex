\subsection{Reprezentacja stanu}
\label{subsec:reprezentacja-stanu}

Reprezentacja stanu zawiera resztę informacji koniecznych do przedstawienia pozycji szachowej odkodowywanej z FEN.
Znajdują się w niej dane o możliwych roszadach, polu en-passant, liczbie posunięć od ostatniego bicia, czy informacji, do którego gracza należy następny ruch.

W tej klasie zawarto także informację o tym, jaki ruch doprowadził do aktualnej pozycji, oraz strukturę HashMap zawierającą informacje, jak często dane pozycja wystąpiły już w grze.
Pozwoliło to na wykrywanie trzykrotnego powtórzenia stanu gry (niekoniecznie następującego bezpośrednio po sobie), które skutkuje automatycznym remisem.

Przechowywanie danych o etapie gry oraz o wyliczonej ocenie heurystycznej pozwoliło na~uniknięcie wielokrotnego przeliczania tych samych wartości przez różne fragmenty programu.


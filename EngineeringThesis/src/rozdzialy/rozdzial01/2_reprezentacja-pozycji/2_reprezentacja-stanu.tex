\subsection{Reprezentacja stanu}
\label{subsec:reprezentacja-stanu}

Reprezentacja stanu zawiera resztę informacji koniecznych do przedstawienia pozycji.
Znajduje się w niej logika dotycząca możliwych roszadach, pola en-passant, czy liczby posunięć od~ostatniego bicia.
Stan gry posiada również wiedzę o ruchu, który do danej pozycji doprowadził, aby ułatwić jego cofanie przez inne komponenty systemu.
W tej klasie zawarto także strukturę HashMap z informacjami, jak często dana pozycja wystąpiła już w grze.

Pozwoliło to na wykrywanie wielokrotnego powtórzenia pozycji, które niekoniecznie następują bezpośrednio po sobie.
Choć według oficjalnych zasad automatyczny remis następuje dopiero po pięciokrotnym powtórzeniu, większość platform do rozgrywek online uznaje trzykrotne powtórzenie za obligatoryjny remis.
Taka też wersja została zaimplementowana.

Reguła remisu dyktowanego pięćdziesięcioma ruchami bez bicia lub ruchu pionem nie~została zaimplementowana.
Ze statystyk wynikło, że takie sytuacje zdarzały się niezwykle rzadko, a gdy się pojawiały, pozycje obydwu graczy można było uznać za równą.
Remis w takiej sytuacji uznano za korzystny dla obydwu graczy.

\subsubsection{Haszowanie Zobrist}

\begin{center}
    \textcolor{red}{\textbf{DODAĆ OPIS}}
\end{center}




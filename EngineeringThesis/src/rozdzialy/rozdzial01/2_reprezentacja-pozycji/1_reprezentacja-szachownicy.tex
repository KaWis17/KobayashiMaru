\subsection{Reprezentacja szachownicy}
\label{subsec:reprezentacja-szachownicy}

Struktury danych, wybrane do reprezentacji szachownicy, w dużym stopniu determinują implementację funkcji generowania dostępnych ruchów, a co za tym idzie, bezpośrednio wpływają na wydajność całego silnika.
Z tego względu ich dobór musiał zostać odpowiednio zaplanowany, uwzględniając ogół architektury programu.
Po zapoznaniu się z proponowanymi w~literaturze rozwiązaniami, w projekcie zdecydowano się zastosować dwie redundantne techniki reprezentacji 64 pól szachowych.

Obie charakteryzują się odmiennymi właściwościami, znajdując zastosowanie dla innych algorytmów wewnątrz programu.
Różnią się one pod względem gęstości zawartych informacji, szybkości dostępu do danych oraz łatwości modyfikacji.
Jedna z nich skupia się na każdym z~pól~szachownicy (ang.~\emph{Square Centric}), druga natomiast bierze pod uwagę położenie konkretnych rodzajów bierek (ang.~\emph{Piece Centric}).

Przy procesie implementacji omawianego fragmentu należało zachować szczególną ostrożność, aby ciągłe aktualizowanie dwóch niezależnych struktur danych nie doprowadziło do~ich rozspójnienia, a co za tym idzie wprowadzenia trudnych do zdebugowania błędów.
Ryzyko ich pojawienia zostało zminimalizowane dzięki wprowadzeniu testów jednostkowych regularnie kontrolujących poprawność operacji podczas enumeracji drzewa gry.
\subsubsection{Tablica pól szachowych}

Naturalnym podejściem do reprezentacji szachownicy wydało się zastosowanie 64 elementowej tablicy, w której każde pole odpowiada konkretnemu miejscu na planszy.
W tej implementacji poszczególne bierki zostały zakodowane liczbami od 1 do 6, natomiast liczby 0 i 8 odpowiednio reprezentują biały oraz czarny kolor.
W ten sposób, za pomocą pojedynczych bajtów, można określić zarówno typ figury, jak i jej kolor na danym polu.

Taka struktura danych jest szczególnie użyteczna, gdy konieczna jest szybka odpowiedź na~pytanie, czy na danym kwadracie znajduje się figura, a jeśli tak, to jaka.
Dzięki prostemu indeksowaniu tablicy dostęp do informacji o stanie pojedynczego pola jest bardzo efektywny.

Wada tej techniki objawia się w momencie, gdy wymagane jest odnalezienie wszystkich pól~zawierających konkretny typ figury.
W takim przypadku konieczna staje się iteracja całej tablicy, w celu zidentyfikowania odpowiednich miejsc.
Operacja ta, szczególnie przy~wielokrotnym wywołaniu, może znacząco obniżyć wydajność implementowanego algorytmu.

%\begin{lstlisting}[
%    language=Java,
%    style=JavaStyle,
%    caption=Implementacja reprezentacji szachownicy tablicą pól,
%    label=lst:pierwszy]
%
%    public byte[] board = new byte[64];
%
%    @Override
%    public void addPieceOnSquare(byte square, byte color, byte piece) {
%        board[square] = (byte) (color | piece);
%    }
%
%    @Override
%    public void deletePieceOnSquare(byte square, byte color, byte piece) {
%        board[square] = 0;
%    }
%
%\end{lstlisting}


\subsubsection{Tablice bitowe bierek}

W celu zaadresowania powyższych spowolnień zastosowana została technika reprezentacji szachownicy za pomocą tablic bitowych, powszechnie znanych w środowisku programistów szachowych jako ang.~\emph{Bitboards}.
Informacja o położeniu konkretnego typu bierki na planszy przechowywana jest w postaci tablicy liczb o rozmiarze 8 bajtów.

Implementacja ta wykorzystuje kodowanie 64 polowej tablicy szachowej w 64 pojedynczych bitach zmiennej.
Oznaczając figurę na danym polu za pomocą jedynki, a pozostałe pola jako zera, można w prosty sposób zawrzeć informacje o planszy w dwunastu pojedynczych literach.
Pozycje bierek nie są natomiast jedyną informacją, którą można zakodować w tablicach bitowych.
Uprzednio policzone maski możliwych ruchów czy atakowanych pól, to tylko niektóre z możliwych zastosowań, które na późniejszych etapach implementacji znacznie przyspieszały obliczenia.

Powszechne użycie 64-bitowej architektury sprawiło, że operacje na danych takiej wielkości są bardzo efektywne, gdyż nie wymagają rozbicia instrukcji na mniejsze części.
Dodatkowym atutem jest także szybkość manipulacji danymi przez operatory bitowe takie jak negacja, koniunkcja, alternatywa wykluczająca czy przesunięcie bitowe.
Z reguły, a w szczególności na starszych procesorach, operacje bitowe są szybsze, niż ich odpowiedniki w postaci operacji arytmetycznych.

%\begin{lstlisting}[
%    language=Java,
%    style=JavaStyle,
%    caption=Implementacja reprezentacji szachownicy tablicami bitowymi bierek,
%    label=lst:drugi]
%    public void addPieceOnSquare(byte square, byte color, byte piece) {
%        bitBoards[piece | color] |= Long.rotateLeft(1L, square-1);
%    }
%
%    public void deletePieceOnSquare(byte square, byte color, byte piece) {
%        bitBoards[piece | color] &= ~Long.rotateLeft(1L, square-1);
%    }
%\end{lstlisting}


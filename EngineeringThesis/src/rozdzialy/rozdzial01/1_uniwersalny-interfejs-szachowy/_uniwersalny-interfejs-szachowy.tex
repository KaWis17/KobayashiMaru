\section{Uniwersalny Interfejs Szachowy}
\label{sec:uniwersalny-interfejs-szachowy}

Aby umożliwić komunikację między programem a innymi aplikacjami szachowymi należy zdefiniować wspólny protokół wymiany informacji.
Standardem, przyjętym nie tylko w silnikach szachowych, ale także w interfejsach graficznych oraz systemach zarządzania rozgrywkami, jest

\subsection{Dane wejściowe}
\label{subsec:dane-wejsciowe}

\subsubsection{Notacja Forsytha-Edwardsa}

\subsubsection{Szachowa notacja algebraiczna}

\subsection{Komunikacja z interfejsu do silnika}
\label{subsec:komunikacja-z-interfejsu-do-silnika}


\subsection{Komunikacja z silnika do interfejsu}
\label{subsec:komunikacja-z-silnika-do-interfejsu}


\subsection{Przykład użycia}
\label{subsec:przyklad-uzycia}


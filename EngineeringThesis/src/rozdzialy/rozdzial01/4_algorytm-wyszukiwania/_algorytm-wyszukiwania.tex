\section{Algorytm wyszukiwania}
\label {sec:algorytm-wyszukiwania}

Z punktu widzenia teorii gier, szachy można zakwalifikować jako dwuosobową grę o sumie zerowej z pełną informacją.
Oznacza to, że każdy z graczy wykonujących ruch ma dostęp do pełnej wiedzy na temat aktualnego stanu gry oraz jej historii.
Co więcej, po wykonaniu ruchu, zysk gracza jest równy stratom poniesionym przez jego oponenta.

W związku ze średnią ilością możliwych posunięć w danej pozycji niemożliwe jest wprowadzenie optymalnej analizy statycznej proponującej następne ruchy.
Z tego względu konieczne jest zastosowanie algorytmów dynamicznie przechodzących przez drzewo gry, w celu znalezienia najlepszego posunięcia.

\subsection{Algorytm min-max \colorbox{yellow}{Implemented}}
\label{subsec:algorytm-min-max}


\subsection{Iteracyjne pogłębianie wyszukiwania}
\label{subsec:iteracyjne-pogebianie-wyszukiwania}

Lorem ipsum dolor sit amet, consectetur adipiscing elit

\subsection{Zarządzanie czasem \colorbox{yellow}{Implemented}}
\label{subsec:zarzadzanie-czasem}
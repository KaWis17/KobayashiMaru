\usepackage{biblatex}\section{Algorytm wyszukiwania}
\label {sec:algorytm-wyszukiwania}

W 1950 roku Amerykański matematyk Claude Shanon opublikował pracę naukową zatytułowaną \("\)Programowanie komputera do gry w szachy\("\)\cite*{Shannon1950XXIIPA}.
Praca ta stała się teoretyczną podstawą tworzenia silników szachowych.
Zawiera ona między innymi oszacowanie co do ilości możliwych pozycji szachowych, wynoszące $10^{43}$.
Choć oszacowanie to zmieniało się w pewnym stopniu na przestrzeni lat, to jednak liczba ta dowodzi jednoznacznie, że z uwagi na skalę problemu nie jest możliwa implementacja tablicy zawierającej wszystkie pozycje szachowe, oraz najlepsze możliwe na nie odpowiedzi.
Koniecznym było zaimplementowanie algorytmu, który dla konkretnej strategii, decydowałby jaki ruch wykonać uwzględniając daną głębokość drzewa decyzyjnego.


\subsection{Algorytm min-max \colorbox{yellow}{Implemented}}
\label{subsec:algorytm-min-max}


\subsection{Iteracyjne pogłębianie wyszukiwania}
\label{subsec:iteracyjne-pogebianie-wyszukiwania}

Lorem ipsum dolor sit amet, consectetur adipiscing elit

\subsection{Zarządzanie czasem \colorbox{yellow}{Implemented}}
\label{subsec:zarzadzanie-czasem}
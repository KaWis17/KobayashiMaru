\newpage
movegeneration
\newpage
\section{Algorytm wyszukiwania}
\label {sec:algorytm-wyszukiwania}

Z punktu widzenia teorii gier, szachy można zakwalifikować jako dwuosobową grę o sumie zerowej z pełną informacją.
Oznacza to, że każdy z graczy wykonujących ruch ma dostęp do pełnej wiedzy na temat aktualnego stanu gry oraz jej historii.
Co więcej, po wykonaniu ruchu, zysk gracza jest równy stratom poniesionym przez jego oponenta.

W związku ze średnią ilością możliwych posunięć w danej pozycji niemożliwe jest wprowadzenie optymalnej analizy statycznej proponującej następne ruchy.
Z tego względu konieczne jest zastosowanie algorytmów dynamicznie przechodzących przez drzewo gry, w celu znalezienia najlepszego posunięcia.

\subsection{Algorytm minimalizowania maksymalnych strat}
\label{subsec:algorytm-minimalizowania-maksymalnych-strat}

Algorytm minimax polega na przeszukiwaniu drzewa gry.
Z uwagi na mnogość możliwych decyzji, proces ogranicza się do określonej głębokości.
Liściom drzewa przypisywane są~obliczone wartości heurystyczne.
Idąc wzwyż grafu, na kolejnych poziomach, nadawane są wartości maksymalne dla protagonisty i minimalne dla antagonisty.
Ruch, który prowadzi z korzenia do wierzchołka na głębokości jeden z największą wartością, jest ruchem proponowanym przez algorytm.

W silniku zastosowano skróconą wersję powyższego algorytmu, wykorzystując fakt, że~$max(\alpha, \beta) = -min(-\alpha, -\beta)$.

\begin{lstlisting}[
    language=Java,
    style=JavaStyle,
    caption=Implementacja algorytmu negaMax,
    label=lst:drugi]
    negaMax(int depth) {
        moves = moveGenerator.generateMoves();
        if(depth == 0 || moves.isEmpty())
                return evaluator.evaluate();

        int bestMoveValue = MINIMUM;
        for(Move move : moves) {
            board.makeMove(move);
            int score = -negaMax(depth-1);
            if(score > bestMoveValue)
                bestMoveValue = score;
            board.unmakeMove();
        }
        return bestMoveValue;
    }
\end{lstlisting}



\subsection{Iteracyjne pogłębianie wyszukiwania \colorbox{yellow}{Implemented}}
\label{subsec:iteracyjne-pogebianie-wyszukiwania}

\input{rozdzialy/rozdzial01/4_algorytm-wyszukiwania/3_zarządzanie-czasem}
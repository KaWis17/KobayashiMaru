
\section{Algorytm wyszukiwania}
\label {sec:algorytm-wyszukiwania}

Z punktu widzenia teorii gier, szachy można zakwalifikować jako dwuosobową grę o sumie zerowej z pełną informacją.
Oznacza to, że każdy z graczy wykonujących ruch ma dostęp do pełnej wiedzy na temat aktualnego stanu gry oraz jej historii.
Co więcej, po wykonaniu ruchu, zysk gracza jest równy stratom poniesionym przez jego oponenta.

W związku ze średnią ilością możliwych posunięć w danej pozycji niemożliwe jest wprowadzenie optymalnej analizy statycznej proponującej następne ruchy.
Z tego względu konieczne jest zastosowanie algorytmów dynamicznie przechodzących przez drzewo gry, w celu znalezienia najlepszego posunięcia.

\subsection{Algorytm minimalizowania maksymalnych strat}
\label{subsec:algorytm-minimalizowania-maksymalnych-strat}

\subsubsection{Minimax Algorithm}

\subsubsection{Negamax Algorithm}

Wywodzący się z teorii gier algorytm min-max jest metodą minimalizacji maksymalnych strat.
Znajduje on szerokie zastosowanie w grach strategicznych o sumie zerowej.

\subsection{Iteracyjne pogłębianie wyszukiwania}
\label{subsec:iteracyjne-pogebianie-wyszukiwania}

Lorem ipsum dolor sit amet, consectetur adipiscing elit
\input{rozdzialy/rozdzial01/4_algorytm-wyszukiwania/3_zarządzanie-czasem}
\subsection{Zarządzanie czasem}
\label{subsec:zarzadzanie-czasem}
\subsubsection{Estymacja dostępnego czasu}
Większość pojedynków szachowych odbywa się w czasie rzeczywistym, z narzuconym ograniczeniem co do łącznego czasu na wykonanie ruchu.
Przekroczenie tego limitu oznacza automatyczną wygraną oponenta.
Koniecznym było zatem zaimplementowanie mechanizmu zarządzania, który pozwoliłby na podział pozostałego czasu, na wykonanie kolejnych ruchów.
Pozostały czas otrzymany przez UCI w formie \texttt{go wtime <wtime> btime <btime> winc <winc> binc <binc>} pozwolił na implementacje rozwiązania opartego na~wzorze:
\begin{align*}
    \text{Est pozostałych ruchów} &= {\max(40 - \text{wykonane ruchy}, 10)} \\
    \text{Est pozostałego czasu} &= \mathrm{time_{color}} + (inc_{color} \cdot \text{Est pozostałych ruchów}) \\
    \text{Propozycja czasu} &= \dfrac{\text{Est pozostałych ruchów}}{\text{Est pozostałego czasu}}
\end{align*}

Założono średnią liczbę ruchów na poziomie czterdziestu.

\subsubsection{Iteratywne pogłębianie wyszukiwania}
Prędkość wykonania algorytmu negaMax zależy od wielu czynników, takich jak głębokość wyszukiwania, aktualna pozycja, prędkość generowania ruchów czy ilość dostępnych dla~programu zasobów.
Nie ma możliwości oszacowania, ile czasu zajmie jego wykonanie, a~przerwanie w trakcie działania, może skutkować wybraniem skrajnie nieopłacalnych posunięć.

W tym celu zaimplementowano rozwiązanie iteracyjnego pogłębiania wyszukiwania, polegającego na szukaniu najlepszego ruchu na kolejnych głębokościach.
W momencie upływu czasu na obliczenia zwracana jest wartość otrzymana z najgłębszego, w pełni ukończonego przeszukiwania.

Takie podejście mogłoby wydawać się nieoptymalne, gdyż wymaga wielokrotnego generowania drzewa gry.
W praktyce jednak jest to rozwiązanie gwarantujące otrzymanie w miarę dobrego ruchu, a ulepszenia algorytmu takie jak dynamiczne sortowanie ruchów pozwalają na osiągnięcie szybszych rezultatów, niż przy przeszukaniu drzewa od razu dla danej głębokości \cite*{wiki-deepening}.
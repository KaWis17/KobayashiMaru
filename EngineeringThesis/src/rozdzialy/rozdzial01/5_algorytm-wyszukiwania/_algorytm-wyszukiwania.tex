\section{Algorytm wyszukiwania}
\label {sec:algorytm-wyszukiwania}


Z punktu widzenia teorii gier, szachy można klasyfikować na wiele różnych sposobów.
\begin{enumerate}
    \item W wydaniu europejskim są one grą dwuosobową.
    \item Jest to gra o sumie stałej, ponieważ w wyniku wykonywania dowolnej liczby ruchów proporcja zysków jednego gracza pozostaje niezmienna w stosunku do strat drugiego.
    Gry o sumie stałej można sprowadzić do gier o sumie zerowej.
    \item Rozgrywka toczy się w postaci ekstensywnej, a więc ruchy wykonywane są na przemian.
    \item Jest to gra skończona, z uwagi na regułę trzykrotnego powtórzenia.
    \item Każdy z graczy posiada dostęp do doskonałej informacji.
\end{enumerate}

Najbardziej istotne, z punktu tworzenia silnika szachowego, jest połączenie dwóch pierwszych z wymienionych cech.
Takie gry nazywane są ściśle konkurencyjnymi.
Innymi słowy, aby uzyskać maksymalną wypłatę, gracz dąży do tego, by zminimalizować sumę wypłat przeciwnika~\cite*{wstep-teoria-gier}.

W algorytmice problemy tego typu można rozwiązać za pomocą algorytmów dążących do~minimalizowania maksymalnych strat.

\subsection{Algorytm minimalizowania maksymalnych strat}
\label{subsec:algorytm-minimalizowania-maksymalnych-strat}

Algorytm minimax polega na przeszukiwaniu drzewa gry.
Z uwagi na mnogość możliwych decyzji, proces ogranicza się do określonej głębokości.
Liściom drzewa przypisywane są~obliczone wartości heurystyczne.
Idąc wzwyż grafu, na kolejnych poziomach, nadawane są wartości maksymalne dla protagonisty i minimalne dla antagonisty.
Ruch, który prowadzi z korzenia do wierzchołka na głębokości jeden z największą wartością, jest ruchem proponowanym przez algorytm.

W silniku zastosowano skróconą wersję powyższego algorytmu, wykorzystując fakt, że~$max(\alpha, \beta) = -min(-\alpha, -\beta)$.

\begin{lstlisting}[
    language=Java,
    style=JavaStyle,
    caption=Implementacja algorytmu negaMax,
    label=lst:drugi]
    negaMax(int depth) {
        moves = moveGenerator.generateMoves();
        if(depth == 0 || moves.isEmpty())
                return evaluator.evaluate();

        int bestMoveValue = MINIMUM;
        for(Move move : moves) {
            board.makeMove(move);
            int score = -negaMax(depth-1);
            if(score > bestMoveValue)
                bestMoveValue = score;
            board.unmakeMove();
        }
        return bestMoveValue;
    }
\end{lstlisting}



\subsection{Zarządzanie czasem}
\label{subsec:zarzadzanie-czasem}
\subsubsection{Estymacja dostępnego czasu}
Większość pojedynków szachowych odbywa się w czasie rzeczywistym, z narzuconym ograniczeniem co do łącznego czasu na wykonanie ruchu.
Przekroczenie tego limitu stanowiłoby automatyczną wygraną oponenta.
Koniecznym było zatem zaimplementowanie mechanizmu zarządzania, który pozwoliłby na podział pozostałego czasu, na wykonanie poszczególnych ruchów.
Pozostały czas otrzymany przez UCI w formie \texttt{go wtime <wtime> btime <btime> winc <winc> binc <binc>} pozwolił na implementacje rozwiązania opartego na~wzorze:
\begin{align*}
    \text{Est pozostałych ruchów} &= {\max(40 - \text{wykonaneRuchy}, 10)} \\
    \text{Est pozostałego czasu} &= time_{color} + (inc_{color} * \text{Est pozostałych ruchów}) \\
    \text{Propozycja czasu} &= \dfrac{\text{Est pozostałych ruchów}}{\text{Est pozostałego czasu}}
\end{align*}

Założono średnią ilość ruchów na poziomie czterdziestu.

\subsubsection{Iteratywne pogłębianie wyszukiwania}
Prędkość wykonania algorytmu negaMax zależy od wielu czynników, takich jak głębokość wyszukiwania, aktualna pozycja, prędkość generowania ruchów czy ilość dostępnych dla~programu zasobów.
Nie ma możliwości oszacowania, ile czasu zajmie jego wykonanie, a~przerwanie w trakcie działania, może skutkować wybraniem skrajnie nieopłacalnych posunięć.

W tym celu zaimplementowano rozwiązanie iteracyjnego pogłębiania wyszukiwania, polegającego na szukaniu najlepszego ruchu dla kolejnych głębokościach.
W momencie upływu czasu na obliczenia zwracana jest wartość otrzymana z najgłębszego, w pełni ukończonego przeszukiwania.

Takie podejście mogłoby wydawać się skrajnie nieoptymalne, gdyż wymaga wielokrotnego generowania drzewa gry.
W praktyce jednak jest to rozwiązanie gwarantujące otrzymanie w miarę dobrego ruchu, a ulepszenia algorytmu takie jak dynamiczne sortowanie ruchów pozwalają na osiągnięcie szybszych rezultatów, niż przy przeszukaniu drzewa od razu dla danej głębokości \cite*{wiki-deepening}.
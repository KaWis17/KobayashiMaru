\section{Algorytm wyszukiwania}
\label {sec:algorytm-wyszukiwania}


Z punktu widzenia teorii gier, szachy można klasyfikować na wiele różnych sposobów.
\begin{enumerate}
    \item W wydaniu europejskim są one grą dwuosobową.
    \item Jest to gra o sumie stałej, ponieważ w wyniku wykonywania dowolnej liczby ruchów proporcja zysków jednego gracza pozostaje niezmienna w stosunku do strat drugiego.
    Gry o sumie stałej można sprowadzić do gier o sumie zerowej.
    \item Rozgrywka toczy się w postaci ekstensywnej, a więc ruchy wykonywane są na przemian.
    \item Jest to gra skończona, z uwagi na regułę trzykrotnego powtórzenia.
    \item Każdy z graczy posiada dostęp do doskonałej informacji.
\end{enumerate}

Najbardziej istotne, z punktu tworzenia silnika szachowego, jest połączenie dwóch pierwszych z wymienionych cech.
Takie gry nazywane są ściśle konkurencyjnymi.
Innymi słowy, aby uzyskać maksymalną wypłatę, gracz dąży do tego, by zminimalizować sumę wypłat przeciwnika. \cite*{wstep-teoria-gier}

W algorytmice problemy tego typu można rozwiązać za pomocą algorytmów dążących do minimalizowania maksymalnych strat.

\subsection{Algorytm minimalizowania maksymalnych strat}
\label{subsec:algorytm-minimalizowania-maksymalnych-strat}

\subsubsection{Minimax Algorithm}

\subsubsection{Negamax Algorithm}

Wywodzący się z teorii gier algorytm min-max jest metodą minimalizacji maksymalnych strat.
Znajduje on szerokie zastosowanie w grach strategicznych o sumie zerowej.

\input{rozdzialy/rozdzial01/5_algorytm-wyszukiwania/2_zarządzanie-czasem}